\documentclass[main.tex]{subfiles}

\begin{document}
\section{Meyerovi valići}\label{sec:meyer}
Meyerovi valići čine jednu od najranije otkrivenih
ortonormiranih valićnih baza s dobrim svojstvima.
Zbog tih svojstava ćemo Meyerove valiće koristiti u zaključnom dijelu ovog rada.
U ovom odjeljku ćemo ih uvesti u kratkim crtama.

\newcommand{\psm}{\psi^{\mathrm{Meyer}}}
\newcommand{\whpsm}{\wh\psi^{\mathrm{Meyer}}}
Definirajmo Meyerov matični valić \( \psm \) preko
\begin{equation}\label{eq:meyer}
	\whpsm(\xi) = \begin{dcases}
		\frac 1{\sqrt{2\pi}}e^{i\xi/2}\sin\left[ \frac \pi2 \nu \left( \frac 3{2\pi} \abs \xi - 1 \right) \right],
		                                                                                                           & \quad \frac {2\pi}3 \le \abs \xi \le \frac {4\pi}3,                                                                   \\
		\frac 1{\sqrt{2\pi}}e^{i\xi/2}\cos\left[ \frac \pi2 \nu \left( \frac 3{4\pi} \abs \xi - 1 \right) \right], &
		\quad \frac {4\pi}3 \le \abs \xi \le \frac {8\pi}3,                                                                                                                                                                                \\
		0,                                                                                                         & \quad                                                                                                   \text{inače},
	\end{dcases}
\end{equation}
gdje je \( \nu \) funkcija neke glatkoće za koju još vrijedi
\begin{equation}
	\nu(x) = \begin{cases}
		0, \quad & x \le 0, \\
		1, \quad & x \ge 1
	\end{cases}
	\qquad \text i \qquad
	\nu(x) + \nu(1-x) = 1, \quad x \in \R.
\end{equation}
Glatkoća od \( \whpsm \) odgovara glatkoći \( \nu \). Neki izbori su \( \nu(x) = x \)
i
\begin{equation}
	\nu(x) = x^4\left( 35-84x+70x^2-20x^3 \right), \quad x \in \R,
\end{equation}
oba klase \( \mrC^\infty \). Sada je dobar trenutak za definirati Schwartzov prostor.

\begin{definicija}
	\emph{Schwartzov prostor} \( \mathcal S(\R) \) je topološki vektorski prostor
	funkcija \( f \in \mrC^\infty(\R) \) sa svojstvom da im svaka derivacija pada brže od bilo kojeg
	inverznog polinoma, tj.\
	\begin{equation}
		\sup_{x \in \R} \abs{x^n f^{(m)}(x)} < \infty, \quad \forall m,n \in \N_0,
	\end{equation}
	a topologija mu je generirana familijom polunormi\footnote{kod polunorme ne mora vrijediti implikacija \( \norm f = 0 \implies f = 0 \)}
	\begin{equation}
		\norm f_{m,n} = \sup_{x \in \R} \abs{x^n f^{(m)}(x)}, \quad m,n \in \N_0.
	\end{equation}
\end{definicija}

Navedimo dvije značajne tvrdnje koje nećemo dokazivati. Prvo, inkluzije
\begin{equation}
	\mrC^\infty_\mrc(\R) \subseteq \mathcal S(\R)
	\qquad \text i \qquad
	\mrC^\infty_\mrc(\R) \subseteq \L^p(\R), \quad 1 \le p < \infty
\end{equation}
su guste, pa je i Schwartzov prostor gust u \( \L^p(\R) \) za \( 1 \le p < \infty \)
unatoč naizgled restriktivnoj definiciji. Drugo, Fourierova transformacija (restrikcija \( \mathcal F \)
iz definicije~\ref{def:four})
je automorfizam Schwartzovog prostora.

U našem slučaju to znači da je \( \whpsm \in \mathcal S(\R) \)
zbog kompaktnog nosača (i ako je \( \nu \) klase \( \mrC^\infty \)).
Onda jer je Fourierova transformacija automorfizam slijedi
i \( \psm \in \mathcal S(\R) \). Po teoremu~\ref{tm:neodredjenosti}
nemoguće je da \( \psm \) također ima kompaktan nosač, tako da je ova situacija
na neki način najbolja moguća. U slučaju da je \( \psi \) ima kompaktan nosač
i \( \wh \psi \in \mathcal S(\R) \) bilo bi nemoguće da je
\( \psi \) klase \( \mrC^\infty \)
po napomeni uz teorem~\ref{tm:daub}.

Potrebno je dokazati da \( \psm \) uopće generira
ortonormiranu bazu na \( \L^2(\R) \).
Prvi dokaz (v.~\cite[\textsection 4.2]{daub}) oslanjao se na neke tehnike
koje nismo obrađivali, a dokaz funkcionira zbog nekih
\enquote{čudesnih} kraćenja. Kasnije se (v.~\cite[\textsection 5.2]{daub})
pokazalo da se do Meyerovih valića može doći i putem MRA. Definira se
skalirajuća funkcija \( \phi \) preko
\begin{equation}
	\wh \phi(\xi) = \begin{dcases}
		\frac 1{\sqrt{2\pi}},
		   & \quad \abs \xi \le \frac {2\pi}3,                                                                           \\
		\frac 1{\sqrt{2\pi}}\cos\left[ \frac \pi2 \nu \left( \frac 3{2\pi} \abs \xi - 1 \right) \right],
		   & \quad  \frac {2\pi}3 \le \abs \xi \le \frac {4\pi}3,                                                        \\
		0, & \quad                                                                                         \text{inače},
	\end{dcases}
\end{equation}
i potprostori \( V_j \) sa
\begin{equation}
	V_j = \olspan \left\{ \phi_{j,k} \st k \in \Z \right\}, \quad j \in \Z.
\end{equation}
Pokazuje se da se \( \wh \psi \) dan sa~\eqref{eq:mradefwhpsi}
poklapa sa \( \whpsm \) iz~\eqref{eq:meyer}.

\end{document}
