\documentclass[main.tex]{subfiles}

\begin{document}
\nocite{*}

\chapter[\holder -regularnost trajektorija FBM]{\holder -regularnost trajektorija frakcionalnog Brownovog gibanja}\label{chapter-gl}
% \chapter{\holder -regularnost trajektorija frakcionalnog Brownovog gibanja}
% \markboth{\holder -regularnost trajektorija FBM}{\holder -regularnost traktorija FBM}
% \markright{\holder -regularnost trajektorija FBM}
\section{Reprezentacija Brownovog gibanja slučajnim redom}\label{sec:gl-brown}
U ovom poglavlju ćemo definirati frakcionalno Brownovo gibanje (FBM), koje je svojevrsna generalizacija
Brownovog gibanja u kojemu ulogu \( 1/2 \) kao parametra samosličnosti zauzima bilo koji drugi
eksponent \( 0 < H < 1 \). Nakon toga ćemo riješiti problem \holder -regularnosti trajektorija FBM
koristeći valiće za razvoj u slučajni red i time pokazati vrijednost harmonijske analize
u primjeni na analizu slučajnih procesa. Pojam \holder -regularnosti uvest ćemo sada.

\begin{definicija}
	Za funkciju \( f \colon \R \to \R \) kažemo da je \emph{\holder -regularna} s parametrom
	\( 0 < \alpha < 1 \) ili da zadovoljava \emph{\holder ov uvjet} ako
	\begin{equation} \label{eq:defholder}
		\abs{f(x)-f(y)} \lesssim \abs{x-y}^\alpha, \quad x,y \in \R.
	\end{equation}
\end{definicija}

Jasno, definiciju je moguće proširiti na funkcije na proizvoljnom metričkom prostoru.
Lako se pokaže da je zahtjev~\eqref{eq:defholder} jači od uniformne neprekidnosti,
ali slabiji od Lipschitz-neprekidnosti (time i diferencijabilnosti). Ideja je da se parametrom
\( \alpha \) kvantificira vrsta glatkoće slabija od diferencijabilnosti. Dobri primjer su funkcije
\( x \mapsto \abs x ^\alpha \) za \( 0 < \alpha < 1 \) --- poznato nam je da te funkcije (korijeni)
nisu diferencijabilne u nuli, ali na njihovim grafovima možemo vidjeti da singularitet u nuli
postaje sve manje izražen povećanjem \( \alpha \). Uvjet~\eqref{eq:defholder} postaje
jači povećanjem \( \alpha \), pa nam je najzanimljiviji najveći \( \alpha \) (supremum) za koji uvjet vrijedi --- takav
\( \alpha \) zove se \emph{kritični \holder ov eksponent}. U slučaju
FBM, vidjet ćemo da parametar \( H \) odgovara tom kritičnom eksponentu. Opet,
za veći eksponent trajektorija ima glađi izgled (v.~\cite[str.~6]{se}).

U ostatku ovog odjeljka prikazat ćemo ukratko motivacijski primjer reprezentacije Brownovog gibanja
pomoću Haarovog sistema.\footnote{budući da ćemo na kraju doći do funkcija u tzv.\ Faber--Schauderovom sistemu, češće se govori o reprezentaciji Faber--Schauderovim sistemom}
Neka je \( \left\{ B_t \st t \in \left[ 0,1 \right] \right\} \) Brownovo gibanje u \( \R \) nad
%\footnote{smisao se nebi promijenio ni na \( \left[ 0,\infty \right\rangle \); spomenimo i još jednu transformaciju na koje je Browonovo gibanje invarijantno: ako stavimo \( Y_0=0 \) i \( Y_t = tX_{t^{-1}} \) za \( t > 0 \) je \( \left\{ Y_t \right\} \) opet Brownovo gibanje}
\( \left[ 0,1 \right] \). Ono ima jednostavni prikaz
pomoću Paley--Wienerovog integrala:
\begin{equation} \label{eq:bmwint}
	B_t = \int_0^1 1_{\left[ 0,t \right]}(s) \D B_s.
\end{equation}
Funkciju \( f = 1_{\left[ 0,t \right]} \) želimo razviti u Haarovom sistemu. Na \( \L^2(\left[0,1  \right]) \)
imamo varijantu Haarovog sistema u odnosu na \( \L^2(\R) \):
\begin{equation}
	h_0 = 1_{\left[0,1 \right]}, \qquad h_{j,k}(s) = 2^{j/2}h(2^js - k), \quad j \ge 0, \ 0 \le k \le 2^j-1
\end{equation}
gdje \( h(s) = 1_{\left[ 0, 1/2 \right\rangle}(s) - 1_{\left[ 1/2,1 \right]}(s) \). Zbog jednostavnog
oblika funkcija možemo direktno izračunati:
\begin{equation}
	\skp f{h_0} = t1_{\left[ 0,1 \right]}, \qquad
	\skp f{h_{j,k}} = 2^{-j/2} \tau(2^jt - k), \quad j \ge 0, \ 0 \le k \le 2^j-1,
\end{equation}
pri čemu je \( \tau \) trokutasta funkcija na \( \left[ 0,1 \right] \) s \( \tau(1/2)=1/2 \), čime
dobivamo
\begin{equation}
	1_{\left[ 0,t \right]}(s) = t1_{\left[ 0,1 \right]}(s) +
	\sum_{j=0}^\infty \sum_{k=0}^{2^j-1}
	2^{-j/2} \tau(2^jt - k)2^{j/2} h(2^js - k).
\end{equation}
Uvrštavanjem i primjenom izometričnosti Paley--Wienerovog integrala dobivamo na\-jav\-lje\-nu reprezentaciju
\begin{equation} \label{eq:brownhaar}
	B_t = t\varepsilon_0 + \sum_{j=0}^\infty\sum_{k=0}^{2^j-1} 2^{-j/2}\tau(2^jt - k) \varepsilon_{j,k},
\end{equation}
pri čemu
\begin{equation}
	\varepsilon_0 = \int_0^1 1_{\left[ 0,1 \right]}(s) \D B_s, \qquad
	\varepsilon_{j,k} = \int_0^1 2^{j/2} h(2^js - k) \D B_s, \quad j \ge 0, \ 0 \le k \le 2^j-1.
\end{equation}
Lako je pokazati \( \varepsilon_0, \varepsilon_{j,k} \sim \mathrm N(0,1) \) i da su nezavisni.
Vrijedi i da red u~\eqref{eq:brownhaar}, osim što konvergira za svaki \( t \) po
normi u \( \L^2(\Omega) \), gotovo sigurno konvergira uniformno po \( t \).
Ovo je ujedno i temelj za \levy jevu konstrukciju Brownovog gibanja. Definiramo li
slučajni proces desnom stranom u~\eqref{eq:brownhaar}, slijede
svojstva \levy jevog procesa (v.~\cite{diez}), a reprezentacija je korisna
i za daljnje bavljenje trajektorijama.

\section[FBM i integralne reprezentacije]{Frakcionalno Brownovo gibanje i integralne reprezentacije}\label{sec:gl-fbm}
Frakcionalno Brownovo gibanje definirat ćemo kao gaussovski proces
s odgovarajućom kovarijacijskom strukturom, a zatim ispitati njegova svojstva.

\begin{definicija}\label{def:fbm}
	Neka je \( H \in \left\langle 0,1 \right\rangle \). \emph{Frakcionalno Brownovo gibanje} (skraćeno FBM ili fBm) s parametrom \( H \) je gaussovki proces \( \left\{ B^H_t \right\} \) u \( \R \) definiran s
	\( \E B^H_t = 0 \) za svaki \( t \ge 0 \) i
	\begin{equation} \label{eq:deffbm}
		\E(B^H_t B^H_s) = \frac 12 \E \left[ (B^H_1)^2 \right]
		\left( t^{2H} + s^{2H} - \abs{t-s}^{2H} \right), \quad t,s\ge 0.
	\end{equation}
\end{definicija}

\noindent Razmotrimo svojstva takvog procesa:
\begin{itemize}
	\item Normalne konačnodimenzionalne distribucije, pa i slučajni proces, jedinstveno su (u smislu distribucije) definirani preko~\eqref{eq:deffbm}. Ipak, potrebno je paziti da odgovarajuće kovarijacijske matrice budu pozitivno semidefinitne. To je moguće dokazati direktno,
	      ili je zbog leme~\ref{lema:sssi} dovoljno dokazati da uopće postoji \( H \)-samosličan
	      proces sa stacionarnim prirastima. Alternativno, FBM se može definirati tzv.\ integralnim
	      reprezentacijama kojima ćemo se baviti vrlo skoro.

	\item Lako se dobije \( B^H_0 = 0 \) g.s. i s tim je~\eqref{eq:deffbm} ekvivalentno sa
	      \begin{equation}\label{eq:deffbmalt}
		      \E \left[ \left( B^H_t-B^H_s \right)^2 \right] = \E\left[ (B^H_1)^2 \right] \abs{t-s}^{2H}.
	      \end{equation}


	\item Varijacija na definiciju~\ref{def:fbm} moguća je ako dodatno pretpostavimo \( \var B_1^H = 1 \) kao npr.\ u~\cite{se}, čime nestaje faktor \( \E\left[ (B_1^H)^2 \right] \) u~\eqref{eq:deffbm}.

	\item Ako uzmemo \( \var B_1^H = 1 \) i \( H=1/2 \), riječ je o običnom Brownovom gibanju. Naime, dobije se \( \E(B^H_tB^H_s) = \min\left\{ t,s \right\} \), što se lako pokaže da vrijedi i za Brownovo gibanje.
	\item Za \( H=1 \) dobio bi se degenerirani slučaj --- kao u dokazu teorema~\ref{tm:sssi} dobili bismo \( \gamma(n) = \E\left[ (B^H_1)^2 \right] \), tj.\ autokorelacija je uvijek \( 1 \) i
	      \( B^H_t = tB_1^H \) g.s.

	\item Za svaki gaussovski \( H \)-samoslični proces sa stacionarnim prirastima
	      nužno vrijedi~\eqref{eq:deffbm} po lemi~\eqref{lema:sssi}. Dakle,
	      ako takav proces postoji je jednak FBM.


	\item Za \( H \neq 1/2 \) prirasti nisu nezavisni po teoremu~\ref{tm:sssi}, što znači da FBM nije \levy jev proces. Štoviše, nije ni aditivan ni Markovljev (v.~\cite{se}).

	\item FBM g.s.\ ima neprekidne i nigdje-diferencijabilne trajektorije. Te tvrdnje
	      slijedit će kad riješimo pitanje \holder -regularnosti trajektorija.

	\item FBM je \( H \)-samoslično. Za \( a > 0 \) vrijedi:
	      \begin{align}
		      \E \left( B^H_{at} B^H_{as} \right) & = \frac 12 \var B^H_1 \left[ (at)^{2H} + (as)^{2H} - (a\abs{t-s})^{2H} \right] \\
		                                          & = a^{2H}\E \left( B^H_t B^H_s \right)
		      = \E \left[ \left( a^H B^H_t \right)\left( a^HB^H_s \right) \right].
	      \end{align}
	      Ponovo jer kovarijacijska struktura određuje gaussovski proces,
	      slijedi \[ \left\{ B^H_{at} \right\} \jpod \left\{ a^HB^H_t \right\}. \]

	\item FBM ima stacionarne priraste. Tvrdnja se opet dobije direktno preko~\eqref{eq:deffbm}. Time smo dokazali sljedeću propoziciju.
\end{itemize}

\begin{propozicija}\label{prop:fbm}
	Za dani \( H \in \left\langle0,1\right\rangle \) je FBM jedinstveni (do na množenje konstantom; u smislu distribucije) centrirani gaussovski \( H \)-samoslični proces sa stacionarnim prirastima.
\end{propozicija}

Sada ćemo prijeći na temu integralnih reprezentacija FBM. Općenito, pod izrazom
integralna reprezentacija podrazumijevamo prikaz slučajnog procesa stohastičkim
integralom nekog drugog procesa (ili determinističke funkcije). U ovom radu
uvijek je riječ o integralu determinističke funkcije s obzirom na Brownovo gibanje --- dakle
o Paley--Wienerovom integralu. Konkretno, želimo prikaz oblika
\begin{equation}\label{eq:intrepgen}
	B^H_t \jpod \int_0^t K(u) \D B_u \quad \text{ili općenitije} \quad
	B^H_t \jpod \int_\R K_t(u) \D B_u
\end{equation}
za \( K, K_t \in \L^2(\R) \) determinističke integralne jezgre. Primijetimo da
u desnoj varijanti~\eqref{eq:intrepgen} trebamo Brownovo gibanje
za \( t \in \R \), što dobijemo npr.\ spajanjem dva nezavisna Brownova gibanja
\( \left\{ B_t \st t \ge 0 \right\} \) i \( \left\{ B_{-t} \st t \ge 0 \right\} \).

Slučajni proces definiran kao u~\eqref{eq:intrepgen} nužno je centriran
i gaussovski. Neka su dani trenuci \( s_1 < t_1 \le s_2 < t_2 \le \cdots \le s_n < t_n \),
realni brojevi \( a_1, \ldots, a_n \) i step-funkcija \( h \) dana s
\[
	h(u) = \sum_{k=1}^n a_k 1_{\left[ s_k, t_k \right]}(u), \quad
	\text{dakle} \quad
	I(h) = \int_\R h(u) \D B_u
	= \sum_{k=1}^n a_k(B_{t_k}-B_{s_k}).
\]
Trivijalna je centriranost tj.\ \( \E I(h) = 0 \). Da je
\( I(h) \) normalno distribuiran slijedi jer je suma nezavisnih
normalnih varijabli iz familije
\(
\left\{ B_{t_k} - B_{s_k} \st k = 1, \ldots, n  \right\}
\)
uz skaliranja s \( a_k \). Nezavisnost i normalnost vrijedi
jer je riječ o prirastima Brownovog gibanja. Oba svojstva
će se prenijeti na opću \( h \in \L^2(\R) \) kao limes
step-funkcija. Zbog dokazanog svojstva, dovoljno je dokazati~\eqref{eq:deffbm}
ili~\eqref{eq:deffbmalt} da bismo dokazali da je s~\eqref{eq:intrepgen}
doista dano FBM (iako, takav dokaz ne govori kako do dotične reprezentacije možemo doći).

Do kraja odjeljka prikazat ćemo dvije takve reprezentacije ---
prva je \emph{Mandelbrot--van~Nessova} ili \emph{(non-anticipative) moving average},
druga se naziva \emph{harmonizabilna} i ona će nam jedina biti potrebna u daljnjem radu.
Još jedna, \emph{Volterrinog tipa} može se naći u~\cite{se}, a
od interesa je jer njena jezgra ima kompaktni nosač.
U literaturi se uvijek dodaju uz integral i konstante potrebne za normalizaciju varijance.
Budući da nam normalizirana varijanca nije bitna, niti ju tražimo
u definiciji~\ref{def:fbm}, konstante
ćemo izostaviti. Ipak, vidljive su iz računa i mogu se naći u~\cite{se}.

\begin{teorem} (Mandelbrot--van Nessova reprezentacija FBM).
	Neka je \( H \in \left\langle 0, 1 \right\rangle \) i\footnote{\( a_+=\max\left\{ 0,a \right\} \)}
	\begin{equation}
		K_t(u) =
		(t-u)_+^{H-1/2} 1_{\left\langle -\infty,0 \right\rangle}(u) - (-u)_+^{H-1/2}, \quad t \ge 0.
	\end{equation}
	Tada je \( \left\{ X_t \right\} \),
	definiran s \( X_t = I(K_t) \), FBM s parametrom \( H \).
\end{teorem}

\begin{proof}
	Neka je bez smanjenja općenitosti \( t > s \ge 0 \).
	Dokazujemo~\eqref{eq:deffbmalt}. Prvu jednakost
	dobivamo zbog linearnosti i izometričnosti Paley--Wienerovog integrala:
	\begin{align}
		\E\left[ \left( X_t-X_s \right)^2 \right]
		 & = \int_\R \left[ (t-u)_+^{H-1/2} - (s-u)_+^{H-1/2} \right]^2 \D u                          \\
		 & = (t-s)^{2H} \int_\R \left[ (u+1)_+^{H-1/2} - (u)_+^{H-1/2}  \right]^2 \D u = C(t-s)^{2H}.
	\end{align}
	Drugu jednakost dobivamo supstitucijom \( u \leftarrow \frac{s-u}{t-s} \)
	nakon čega \( (t-s)^{2H} \) izlazi van. Preostaje samo konstatirati da je
	integral koji stoji iza \( C \) doista konačan.
\end{proof}

\begin{teorem} (Harmonizabilna reprezentacija FBM).
	Neka je \( H \in \left\langle0,1\right\rangle \) i
	\begin{equation}
		K_t(u) =
		\abs{u}^{-H-1/2} \cdot \begin{cases}
			\sin tu, \quad   & u \ge 0, \\
			1-\cos tu, \quad & u < 0.
		\end{cases}
	\end{equation}
	Tada je \( \left\{ X_t \right\} \),
	definiran s \( X_t = I(K_t) \), FBM s parametrom \( H \).
\end{teorem}

\begin{proof}
	Počinjemo kao u prošlom dokazu:
	\begin{align}
		\E \left[ \left( X_t-X_s \right)^2 \right] & =
		\int_0^\infty \frac{(\sin tu - \sin su)^2}{u^{2H+1}} \D u +
		\int_{-\infty}^0 \frac{(\cos tu - \cos su)^2}{(-u)^{2H+1}} \D u                                  \\
		                                           & = \int_0^\infty \frac{2-2\cos(t-s)u}{u^{2H+1}} \D u \\ &=  (t-s)^{2H}\int_0^\infty \frac{1-\cos u}{u^{2H+1}} \D u =C(t-s)^{2H}.
	\end{align}
	Pritom, druga jednakost dobiva se supstitucijom \( u \leftarrow -u \)
	u drugom integralu da se svedu na jedan, zatim se
	dovršava s \( u \leftarrow (t-s)u \). Očito je \( C \) opet konačan.
\end{proof}

\begin{komentar}
	Bla bla
\end{komentar}

\end{document}
