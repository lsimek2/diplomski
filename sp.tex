\documentclass[main.tex]{subfiles}

\begin{document}
\nocite{*}

\chapter{Slučajni procesi u neprekidnom vremenu} \label{ch:sp}
\section{Osnovni pojmovi} \label{sec:sp-osnovni}
Slučajni proces je indeksirana familija slučajnih veličina \( \left\{ X_t \st t \in T \right\} \) na istom vjerojatnosnom prostoru
gdje je \( T \) neki indeksni skup. On može biti prebrojiv (npr.\ \( T = \N \) ili \( T = \Z \))
ili neprebrojiv (npr.\ \( T = \R \) ili \( T = \left[ 0, \infty \right]  \)). Budući da o
elementima \( T \) često razmišljamo kao o vremenskim trenucima, govorimo o procesu u diskretnom
ili u neprekidnom vremenu.
U kontekstu ovog rada podrazumijevamo \( T = \left[ 0,\infty \right] \) i koristimo skraćenu notaciju
\( \left\{ X_t \right\} \).

Razlikujemo proces od realizacije
\[ t \mapsto X_t(\omega), \quad t \in \left[ 0,\infty \right], \quad \mathrm{za \ neki} \ \omega \in \Omega \]
koju još zovemo \emph{trajektorijom}. Nadalje, kažemo da je slučajni proces \( \left\{ X_t \right\} \) \emph{modifikacija} procesa
\( \left\{ Y_t \right\} \) ako je \( \P(X_t = Y_t) = 1 \) za sve \( t \in \left[ 0,\infty \right] \). Primijetimo razliku u odnosu
na jači uvjet \( \P(X_t = Y_t,\ \forall t \in \left[ 0,\infty \right]) = 1 \) kada procese više uopće ne razlikujemo.

Slučajne procese obično promatramo preko konačnodimenzionalnih funkcija distribucije
\( F_{t_1, \ldots, t_n} \), \( t_j \in T \), koje zadovoljavaju tzv.\ uvjete suglasnosti Kolmogorova.
Dva su važna rezultata koja to opravdaju --- teorem Kolmogorova (v.~\cite[tm.\ 9.6.]{sarapa}) koji tvrdi
da suglasna familija konačnodimenzionalnih distribucija inducira je\-din\-stve\-nu vjerojatnosnu mjeru na \( \sigma \)-algebri cilindara
i drugi teorem (v.~\cite[tm.\ 9.7.]{sarapa}) koji tvrdi da se za takvu familiju uvijek može konstruirati
slučajni proces (odn.\ vjerojatnosni prostor) s upravo tim konačnodimenzionalnim distribucijama. Stoga nam, kao i obično,
nije bitan vjerojatnosni prostor u kojem se nalazimo.

Sada ćemo definirati \levy jeve procese, jednu od najznamenitijih klasa slučajnih procesa. Uvest ćemo Poissonov proces
i Brownovo gibanje kao poznate primjere. Pokazat ćemo i vrlo blisku vezu \levy jevih procesa i beskonačno djeljivih distribucija (definiranih u~\ref{unkno}), koja
je potrebna za naš dokaz da Brownovo gibanje (i \levy jevi procesi općenito) postoji. Nakon toga ćemo skrenuti pažnju na teoriju samosličnih procesa i uvesti
frakcionalno Brownovo gibanje kao generalizaciju. Prije te definicije, uvedimo još jedan osnovni pojam.

\begin{definicija}
	Slučajni proces \( \left\{ X_t \right\} \) je \emph{stohastički neprekidan} ako vrijedi
	\begin{equation} \label{def:stohnep}
		\lim\limits_{s \rightarrow t} \P (\abs{X_s - X_t} > \varepsilon)  = 0,
	\end{equation}
	za svaki \( t \ge 0 \) i \( \varepsilon > 0 \).
\end{definicija}

\begin{definicija}
	Slučajni proces \( \left\{ X_t \right\} \) u \( \R^d \) je \emph{\levy jev} ako vrijede uvjeti:
	\begin{enumerate}[label=(\roman*)]
		\item \( X_0 = 0 \) g.s.,
		\item \( \left\{ X_t \right\} \) je stohastički neprekidan,
		\item varijable \( X_{t_0}, X_{t_1}-X_{t_0}, \ldots ,X_{t_n}-X_{t_{n-1}} \) su nezavisne za sve \( n \ge 1 \) i \(0 \le t_1 < t_2 < \cdots < t_n \) (nezavisnost inkremenata),
		\item distribucija varijable \( X_{s+t}-X_s \) ne ovisi o \( s \) (stacionarnost inkremenata), \label{stacink}
		\item Postoji \( \Omega_0 \in \mathcal F \) takav da je \( \P(\Omega_0) = 1 \) te za svaki \( \omega \in \Omega_0 \) vrijedi da je trajektorija
		      \( t \mapsto X_t(\omega) \) \cadlag\footnote{\cadlag, od fr.\ \textit{continue à droite, limite à gauche}, označava funkciju koja je na cijeloj domeni neprekidna zdesna s limesima slijeva}. \label{cadlag}
	\end{enumerate}
	Nadalje, ako ne vrijedi~\ref{stacink} kažemo da je proces \emph{aditivan}. Ako ne vrijedi~\ref{cadlag} tada kažemo da je proces \emph{\levy jev po distribuciji} odn.\
	\emph{aditivan po distribuciji}.
\end{definicija}

\section{Poissonov proces}\label{sec:sp-poisson}
U ovom odjeljku uvodimo Poissonov proces kao primjer \levy jevog procesa. Intuitivno, taj proces mjeri broj doagađaja koji su se dogodili do
trenutka \( t \), pri čemu su duljine intervala između uzastopnih događaja distribuirane eksponencijalno. Pritom, eksponencijalnu distribuciju karakterizira
tzv.\ gubitak pamćenja. Poissonov proces uvest ćemo kao \levy jev proces s dodatnim svojstvom da je \( X_t \sim \mathrm P(\lambda t) \), a onda konstrukcijom
dokazati da takav proces zaista postoji. Kasnije ćemo pokazati da je moguće na taj način odrediti \levy jev proces i kad se Poissonova razdioba zamijeni
bilo kojom beskonačno djeljivom distribucijom.

\begin{definicija}
	Slučajni proces \( \left\{ X_t \right\} \) na \( \R \) je \emph{Poissonov proces s parametrom \( \lambda > 0\)} ako je \levy jev i ako
	\begin{equation} \label{eq:poispois}
		X_t \sim \mathrm{P}(\lambda t), \quad t > 0.
	\end{equation}
\end{definicija}

\begin{teorem}
	Neka je \( (W_n)_{n \ge 0} \) slučajna šetnja na \( \R \) definirana na vjerojatnosnom prostoru
	\( (\Omega, \mathcal F, \P) \), takva da \( T_n = W_n - W_{n-1} \) ima eksponencijalnu distribuciju
	s parametrom \( \lambda > 0 \). Definiramo
	\begin{equation}\label{eq:poiscons}
		X_t(\omega) = n \iff W_n(\omega) \le t < W_{n+1}(\omega).
	\end{equation}
	Tada je \( \left\{ X_t \right\} \) Poissonov proces s parametrom \( \lambda \).
\end{teorem}

\begin{proof}
	Ustanovimo prvo da \( W_n \sim \Gamma(n, \lambda) \). Naime, \( T_n \sim \mathrm{Exp}(\lambda) \equiv \Gamma(1, \lambda) \) i
	\( W_n = \sum_{j=1}^n T_j \) pa tvrdnja slijedi po poznatoj lemi o zbroju nezavisnih \( \Gamma \)-distribuiranih varijabli\footnote{zbroj
		je \( \Gamma \)-distribuiran, a prvi parametar (parametar oblika) je zbroj prvih parametara sumanada}. Sada slijedi \( \P(W_n \le t) \rightarrow 0,\ n \ub \), što se
	može dobiti i iz
	\[
		\P(W_n \le t) \le \P(T_1 \le t, T_2 \le t, \ldots, T_n \le t) = \left[ \P(T_1 \le t) \right]^n \rightarrow 0, \textquotestraightdblbase n \ub.
	\]
	Slijedi da je svaka \( X_t \) g.s.\ dobro definirana preko~\eqref{eq:poiscons}. Da je \( \left\{ X_t \right\} \) stohastički neprekidan
	s \cadlag \ trajektorijama i \( X_0 = 0 \) g.s.\ je trivijalno. Preostaje dokazati~\eqref{eq:poispois} i da su inkrementi stacionarni i nezavisni.
	
	Prvo se dobije iz
	\begin{align}
		\P(X_t = n) & = \P(W_n \le t < W_n + T_{n+1}) =
		\int_{\R^2} 1_{\left\{ (u, v) \in \R^2 \st 0 \le u \le t < u+v \right\}}(x,y) \D \P_{(W_n, T_{n+1})} \\ &= \frac{\lambda^{n+1}}{(n-1)!} \int_0^t x^{n-1}e^{-\lambda x} \int_{t-x}^\infty e^{-\lambda y} \D y \D x
		= \frac{(\lambda t)^n}{n!}e^{-\lambda t},
	\end{align}
	pri čemu koristimo nezavisnost varijabli \( W_n \) i \( T_{n+1} \) za prelazak na njihove gustoće. Sličnim izravnim računom možemo dobiti i
	\begin{equation} \label{eq:poiswnuzxt}
		\P(W_{n+1} > t + s \given X_t = n) = e^{-cs} = \P(T_1 > s), \quad t > 0, s \ge 0, n \ge 0.
	\end{equation}
	Pomoću toga možemo dobiti da \( (W_{n+1} - t, T_{n+2}, \ldots, T_{n+m}) \) uz dano \( X_t = n \)
	ima jednaku distribuciju kao i \( (T_1, T_2, \ldots, T_m) \). Vrijedi:
	\begin{align}
		\P & (W_{n+1} -t > s_1 , T_{n+2} > s_2, \ldots, T_{n+m} > s_m \given X_t = n)         \\
		   & =\P(W_n \le t, W_{n+1} -t > s_1, T_{n+2} > s_2, \ldots, T_{n+m} > s_m)/\P(X_t=n) \\
		   & =\P(W_n \le t, W_{n+1} -t > s_1)\P(T_{n+2}>s_2,\ldots,T_{n+m}>s_m)/\P(X_t=n)     \\
		   & =\P(W_{n+1} -t>s_1 \given X_t = n)\P(T_{n+2}>s_2,\ldots,T_{n+m}>s_m)             \\
		   & =\P(T_1>s_1)\P(T_2>s_2,\ldots,T_m>s_m)                                           \\
		   & =\P(T_1>s_1,T_2>s_2,\ldots,T_m>s_m). \label{eq:poisuvj}
	\end{align}
	To nam zatim daje
	\begin{equation}
		\begin{aligned} \label{eq:poislema}
			\P & (W_{n+m} \le t+s < W_{n+m+1} \given X_t = n)                                                   \\
			   & =\P((W_{n+1}-t)+T_{n+2} + \cdots + T_{n+m} \le s < (W_{n+1}-t) + T_{n+2} + \cdots + T_{n+m+1}) \\
			   & \overset{\footnotesize\eqref{eq:poisuvj}}= \P(T_1+\cdots+T_m \le s < T_1+\cdots+T_m+T_{m+1})
			= \P(W_m \le s < W_{m+1}) = \P(X_s=m).
		\end{aligned}
	\end{equation}
	Sada stacionarnost inkremenata slijedi sumiranjem po \( n \ge 0 \) jednakosti:
	\begin{equation}
		\begin{aligned}
			\P & (X_{t+s}-X_t =  m, X_t=n) = \P(X_{t+s}=n+m, X_t=n)                              \\
			   & = \P(X_t=n)\P(W_{n+m} \le t+s < W_{n+m+1} \given X_t = n) = \P(X_t=n)\P(X_s=m).
		\end{aligned}
	\end{equation}
	
	Nezavisnost inkremenata slijedi iz
	\begin{equation}
		\begin{aligned}
			\P & (X_{t_0}=n_0, X_{t_1}- X_{t_0} = n_1, \ldots, X_{t_k}-X_{t_{k-1}} = n_k)                                \\
			   & =\P(X_{t_0} = n_0, X_{t_1}=n_0+n_1, \ldots, X_{t_k} = n_0+\cdots+n_k \given X_{t_0}=n_0)\P(X_{t_0}=n_0) \\
			   & =\P(X_{t_1-t_0}=n_1,\ldots,X_{t_k-t_0}=n_1+\cdots+n_k)\P(X_{t_0}=n_0)
		\end{aligned}
	\end{equation}
	gdje druga jednakost slijedi primjenom iste ideje kao u dokazu stacionarnosti. Dokaz se završi indukcijom.
\end{proof}

\section{Brownovo gibanje}\label{sec:sp-brownovo}


\end{document}
