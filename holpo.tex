\documentclass[main.tex]{subfiles}

\begin{document}
\nocite{*}

\section{\holder -regularnost u točkama}\label{sec:holpo}
%fiksirati prozivoljni Hi
%napomenuti da je \psi meyerov mat. valic
Potrebno je definirati kritični \emph{točkovni} \holder ov eksponent.

\begin{definicija}
	Neka je \( \left\{ X_t \right\} \) slučajni proces čije su trajektorije gotovo sigurno
	lokalno ograničene i nigdje diferencijabilne. Za proizvoljni
	\( t_0 \ge 0 \) definira se \emph{kritični točkovni \holder ov eksponent} procesa
	\( \left\{ X_t \right\} \) u \( t_0 \) sa
	\begin{equation}
		\alpha(t_0) = \sup \left\{ \alpha > 0 \st
		\limsup\limits_{h \rightarrow 0} \frac{\abs{X_{t_0+h}-X_{t_0}}}{\abs h^\alpha} = 0
		\right\}.
	\end{equation}
\end{definicija}

%glavni teorem je 3.9 u ayache
%napomenuti da je potrebno dokazati \le dio
%\Omega_1 ne smije ovisiti o t_0

Prije nego nastavimo, još je korisno definirati dva pojma
standardna u frakcionalnom kalkulusu.

\begin{definicija}
	%Neka je \( \psi \) Meyerov matični valić.
	\begin{enumerate}
		\item \emph{Lijeva frakcionalna antiderivacija reda} \( H+1/2 \) od \( \psi \) definira se sa
		      \begin{equation}
			      \Psi_H(x) = \int_\R e^{ix\xi} \frac{\wh \psi(\xi)}{(i\xi)^{H+1/2}} \D \xi, \quad x \in \R.
		      \end{equation}
		\item \emph{Desna frakcionalna derivacija reda} \( H+1/2 \) od \( \psi \) definira se sa
		      \begin{equation}
			      \Psi_{-H}(x)= \int_\R e^{ix\xi}\wh \psi(\xi)(-i\xi)^{H+1/2} \D \xi, \quad x \in \R.
		      \end{equation}
	\end{enumerate}
\end{definicija}

\begin{komentar}
	...
\end{komentar}

Konačno, dolazimo do valićne reprezentacije FBM.

\begin{teorem}
	Neka je FBM dano s (komentar~\ref{kom:harmrep})
	\begin{equation}
		B^H_t = \int_\R \frac{e^{it\xi}-1}{(i\xi)^{H+1/2}} \D \wh B_\xi, \quad t \ge 0
	\end{equation}
	i
	\( \left\{ \varepsilon_{j,k} \st j,k \in \Z \right\} \)
	nezavisne jednako distribuirane varijable iz \( \mathrm N(0,1) \) definirane s
	\begin{equation}\label{eq:defepsjk}
		\varepsilon_{j,k} =
		\int_\R 2^{-j/2}e^{ik2^{-j}\xi}\wh \psi(-2^{-j}\xi)\D \wh B_\xi =
		\int_\R 2^{j/2} \psi(2^jt-k) \D B_t, \quad j,k\in \Z.
	\end{equation}
	Tada \( \left\{ B^H_t  \right\} \) ima reprezentaciju
	\begin{equation}\label{eq:valfbmrep}
		B^H_t = \sum_{j,k \in \Z}
		2^{-jH} \left( \Psi_H(2^jt - k) - \Psi_H(-k)  \right) \varepsilon_{j,k}, \quad t \ge 0
	\end{equation}
	pri čemu konvergencija vrijedi i g.s.\ uniformno po \( t \) na svakom
	kompaktnom podskupu od \( \R \).
\end{teorem}

\begin{proof}
	Definirajmo prvo  \( \left\{ \varepsilon_{j,k }\right\} \) desnom stranom~\eqref{eq:defepsjk}, tj.\ \( \varepsilon_{j,k} = I(\psi_{j,k}) \).
	Po diskusiji iz odjeljka~\ref{sec:fbm}, znamo da su varijable \( \left\{ \varepsilon_{j,k} \right\} \)
	centrirane i normalno distribuirane. Varijancu dobijemo zbog izometričnosti
	Paley--Wienerovog integrala:
	\begin{equation}
		\E\varepsilon_{j,k}^2 = \E(I(\psi_{j,k})^2) = \norm{\psi_{j,k}}^2_{\L^2} = 1.
	\end{equation}
	Na isti način možemo dobiti nekoreliranost: \( \E (\varepsilon_{j,k}\varepsilon_{j',k'}) = \skp{\psi_{j,k}}{\psi_{j',k'}} = 0 \) za \( (j,k) \neq (j',k') \). Ako varijable \( \left\{ \varepsilon_{j,k}  \right\} \)
	zajedno čine normalne slučajne vektore, tada je to dovoljno za nezavisnost.
	To je istina --- možemo nastaviti razmatranja iz odjeljka~\ref{sec:fbm} pa dokazati i ovo: ako su \( f_1, \ldots, f_n \in \L^2(\R) \)
	linearno nezavisne, tada je \( \left( I(f_1),\ldots,I(f_n) \right)  \) normalni slučajni vektor.

	Slijedeći pravila za modulacije i translacije (\ref{??})
	dobiva se da je Fourierova transformacija od \( \psi_{j,k} \) jednaka
	\( \xi \mapsto -2^{j/2}e^{ik2^{-j}\xi}\wh\psi(2^{-j}\xi) \).
	Konjugacijom, jer je Meyerov valić realan,
	dobiva se integrand u sredini~\eqref{eq:defepsjk}.
	Budući da je integral realan, konjugaciju dodamo bez promjene i time
	je~\eqref{eq:defepsjk} dokazano.

	Jer je \( \left\{ \psi_{j,k} \st j,k \in \Z \right\} \) ortonormirana baza
	prostora \( \L^2(\R) \) i Fourierova transformacija izometrija, familija
	\[
		\left\{ \xi \mapsto 2^{-j/2} e^{ik2^{-j}\xi}\wh\psi(-2^{-j}\xi) \st j,k \in \Z \right\}
	\]
	je također ortonormirana baza prostora \( \L^2(\R) \). Sada želimo dekomponirati
	funkciju \( f(\xi) = \frac{e^{it\xi}-1}{(i\xi)^{H+1/2}} \), koja je očito u \( \L^2(\R) \), u toj bazi:
	\begin{equation} \label{eq:dekompfxi}
		f(\xi) =
		\sum_{j,k \in \Z} c_{j,k}(t) 2^{-j/2}e^{ik2^{-j}\xi}\wh \psi(-2^{-j}\xi),
	\end{equation}
	gdje
	\begin{equation}
		\begin{aligned}
			c_{j,k}(t) & = 2^{-j/2} \int_\R \frac{e^{it\xi}-1}{(i\xi)^{H+1/2}}e^{-ik2^{-j}\xi}\wh \psi(2^{-j}\xi) \D \xi
			\\ &= 2^{-jH} \left( \Psi_H(2^jt-k)-\Psi_H(-k) \right),
		\end{aligned}
	\end{equation}
	a zadnja jednakost dobiva se supstitucijom \( \xi \leftarrow 2^{-j}\xi \).
	Integriranjem~\eqref{eq:dekompfxi} po \( \xi \), zamjenom sume i integrala (dominirana konvergencija zbog \( \Psi_H, \widehat{\psi} \in \mathcal S(\R) \)) i izometričnosti Paley--Wienerovog integrala
	dobiva se~\eqref{eq:valfbmrep}.

	Time smo dokazali da konvergencija u~\eqref{eq:valfbmrep} vrijedi u \( \L^2(\Omega) \), ali ne i g.s.\ uniformno po \( t \) na kompaktu. Dokaz te tvrdnje oslanja
	se na dva teorema izvan opsega ovog rada i može se naći u~\cite[tm.~3.15]{ayache}.

\end{proof}

Slijedi zanimljiva lema koja nam
omogućava da cijelu familiju slučajnih varijabli
ocijenimo jednom zajedničkom. Iz te perspektive
ju možemo zvati slučajnom konstantom.
Dokaz je adaptiran iz~\cite[str~459.]{jourfa-ayache}.
Primijetimo da tražimo da su varijable jednako distribuirane,
ali ne i da su nezavisne.

\begin{lema}
	Neka je \( \left\{ \varepsilon_{j,k} \right\} \) familija jednako distribuiranih
	\( \mathrm N(0,1) \) varijabli. Tada postoji \( \Omega_1^* \) s \( \P(\Omega_1^*) = 1 \)
	takav da
	\begin{equation}
		\abs{\varepsilon_{j,k}(\omega)} \le C_1(\omega) \sqrt{\log(2 + \abs j + \abs k)}
		,\quad \omega \in \Omega_1^*, \ j,k \in \Z.
	\end{equation}
	gdje je \( C_1  \) slučajna varijabla čiji je svaki moment konačan.
\end{lema}

\begin{proof}
	Gledajmo prvo \( \N \)-indeksiranu familiju \( \left\{ \varepsilon_n \right\} \)
	i dokažimo analognu tvrdnju
	\begin{equation}\label{eq:holpo-lema1-anal}
		\abs{\varepsilon_n(\omega)} \le C(\omega) \sqrt{\log(2+n)}.
	\end{equation}
	Koristeći ocjenu iz dokaza teorema~\ref{tm:aditgauss} je:
	\begin{equation}
		\sum_{n=1}^\infty \P\left(\abs{\varepsilon_n} > a\sqrt{\log(2+n)}\right) \lesssim
		\sum_{n=1}^\infty (2+n)^{-a^2/2},
	\end{equation}
	što konvergira ako odaberemo \( a > \sqrt 2 \). Po Borel--Cantellijevoj lemi
	\begin{equation}
		\P\left(\abs{\varepsilon_n} > a\sqrt{\log(2+n)} \text{\ za beskonačno mnogo } n \in \N \right) = 0
	\end{equation}
	pa za \( \Omega_1^* \) stavimo komplement tog događaja, tj.\ na \( \Omega_1^* \)
	je dobro definirano vrijeme zaustavljanja
	\begin{equation}
		T(\omega) = \min \left\{ n \in \N \st m \ge n \implies
		\abs{\varepsilon_m(\omega)} \le a \sqrt{\log(2+m)}  \right\}.
	\end{equation}
	Ideja je sada \( C \) odabrati tako da u~\eqref{eq:holpo-lema1-anal} pokrije
	onih konačno \( n \) za koje tvrdnja već ne vrijedi. Nije očito da takva varijabla
	ima sve momente konačne. Ipak, možemo gledati i veću slučajnu konstantu
	\begin{equation}
		\widetilde C(\omega) = \max\limits_{0 \le m \le T(\omega)} \abs{\varepsilon_m(\omega)},
		\quad \omega \in \Omega_1^*
	\end{equation}
	i dokazati da su joj svi momenti konačni kako slijedi
	\begin{align}
		\E \widetilde C^p & =
		\sum_{n=1}^\infty \E \left[ \max\limits_{0 \le m \le T} \abs{\varepsilon_m}^p 1_{\left\{ T=n \right\}} \right]                                                  \\
		                  & \le \E \left[ \abs{\varepsilon_1}^p 1_{\left\{T=1\right\}}  \right]
		+ \sum_{n=2}^\infty \sum_{m=1}^n \E\left[ \abs{\varepsilon_m}^p 1_{\left\{ \abs{\varepsilon_{n-1}} > a\sqrt{\log(n+1)} \right\} } \right]                       \\
		                  & \le \E \abs{\varepsilon_1}^p + \sum_{n=1}^\infty \sum_{m=1}^n
		\sigma_{2p}^{1/2} \P \left( \abs{\varepsilon_{n-1}} > a\sqrt{\log(n+1)} \right)^{1/2}                                                                           \\
		                  & \lesssim \E \abs{\varepsilon_1}^p + \sigma_{2p}^{1/2}\sum_{n=2}^\infty (n+1)^{-a^2/4+1} < \infty \quad \left(\text{za } a > 2\sqrt2\right).
	\end{align}
	Pritom, druga nejednakost se dobije \holder ovom nejednakosti, uz oznaku
	\( \sigma_{2p} = \E \abs{\varepsilon_m}^{2p} \) (ne ovisi o \( m \)).
	Za treću primjenimo istu ocjenu repa normalne distribucije kao prije i koristimo
	jednaku distribuiranost.

	Jedan način za prelazak na \( (\Z\times\Z) \)-indeksiranu familiju je sljedeći:
	uzmemo bijekciju \( f\colon \Z\times\Z \to \N \) takvu da
	\[
		n = f(j,k) \implies n \lesssim \abs j + \abs k.
	\]
	To svojstvo ima uobičajena bijekcija (v.~\cite[lema~2]{jourfa-ayache} za eksplicitnu definiciju).
	Traženo se dobije iz~\eqref{eq:holpo-lema1-anal} uz odgovarajuću izmjenu konstante.

\end{proof}

\end{document}
