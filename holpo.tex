\documentclass[main.tex]{subfiles}
\begin{document}
\nocite{*}

\section{\holder -regularnost po točkama}\label{sec:holpo}
Definirat ćemo kritični \emph{točkovni} \holder ov eksponent.
Cilj ovog odjeljka je dokazati da je i takav eksponent g.s.\ jednak \( H \). Odsad ćemo promijeniti konvenciju glede FBM~---~trenutke
odsad čine \( t \in \R \) umjesto \( t \ge 0 \), tj.\ govorimo
o procesu \( \left\{ B^H_t \st t \in \R \right\} \).
To možemo napraviti ako~\eqref{eq:deffbm} tražimo
za sve \( t,s \in \R \) umjesto \( t,s \ge 0 \) ili
naprosto proširivanjem integralnih reprezentacija na \( t \in \R \).
Time ćemo se uskladiti s~\cite{ayache} i kasnije
postići veću elegantnost.

Uvodi se još jedna promjena konvencije, naime je \( \psi_{j,k}(x) = 2^{j/2}\psi(2^jx-k) \). Dakle, riječ je o dilataciji inverznoj onoj u~\eqref{eq:psijk}.
Nadalje pod \( \psi = \psi^{\mathrm{Meyer}}\) podrazumijevamo
Meyerov matični valić (v.~odjeljak~\ref{sec:meyer}).


\begin{definicija}\label{def:holpo}
	Neka je \( \left\{ X_t \right\} \) slučajni proces čije su trajektorije gotovo sigurno
	lokalno ograničene i nigdje diferencijabilne. Za proizvoljni
	\( t_0 \in \R \) definira se \emph{kritični točkovni \holder ov eksponent} procesa
	\( \left\{ X_t \right\} \) u \( t_0 \) kao slučajna varijabla
	\begin{equation}
		\alpha_{t_0} = \sup \left\{ \alpha > 0 \st
		\limsup\limits_{h \rightarrow 0} \frac{\abs{X_{t_0+h}-X_{t_0}}}{\abs h^\alpha} = 0
		\right\}.
	\end{equation}
\end{definicija}

%glavni teorem je 3.9 u ayache
%napomenuti da je potrebno dokazati \le dio
%\Omega_1 ne smije ovisiti o t_0
Primijetimo da trebamo samo dokazati \( \alpha_{t_0} \le H \)
jer za suprotnu nejednakost znamo da vrijedi po prošlom odjeljku (za odgovarajuću
modifikaciju, koju podrazumijevamo).
Prije nego nastavimo, još je korisno definirati dva pojma
standardna u frakcionalnom diferencijalnom i integralnom računu.

\begin{definicija}
	%Neka je \( \psi \) Meyerov matični valić.
	\begin{enumerate}
		\item \emph{Lijevi frakcionalni integral reda} \( H+1/2 \) od \( \psi \) definira se sa
		      \begin{equation}
			      \Psi_H(x) = \int_\R e^{ix\xi} \frac{\wh \psi(\xi)}{(i\xi)^{H+1/2}} \D \xi, \quad x \in \R.
		      \end{equation}
		\item \emph{Desna frakcionalna derivacija reda} \( H+1/2 \) od \( \psi \) definira se sa
		      \begin{equation}
			      \Psi_{-H}(x)= \int_\R e^{ix\xi}\wh \psi(\xi)(-i\xi)^{H+1/2} \D \xi, \quad x \in \R.
		      \end{equation}
	\end{enumerate}
\end{definicija}

\begin{komentar}\label{kom:Psi}
	O frakcionalnom integriranju i deriviranju može se naći u~\cite{fidta}.
	Posebice, u~\cite[\textsection 7]{fidta} dokazuje se da vrijede frakcionalni analogoni
	rezultata iz komentara~\ref{kom:four-deri}. Nama će samo biti bitna
	posljedica \( \wh \Psi_{\pm H}(0)=0 \).
	Ostala svojstva \( \Psi_{\pm H} \) slijede iz odgovarajućih svojstava Meyerovih valića, pa
	su tako \( \Psi_{\pm H} \) dobro definirane, pripadaju Schwartzovom prostoru
	i poprimaju samo realne vrijednosti.
\end{komentar}

U sljedećem teoremu konačno opet susrećemo teoriju valića na tzv.\ valićnoj
reprezentaciji FBM, iako će direktniju ulogu imati \( \Psi_H \).
Ta reprezentacija bit će temelj za našu daljnju analizu.

\begin{teorem}\label{holpo-teorem1}
	Neka je FBM dano s (komentar~\ref{kom:harmrep})
	\begin{equation}
		B^H_t = \int_\R \frac{e^{it\xi}-1}{(i\xi)^{H+1/2}} \D \wh B_\xi, \quad t \in \R
	\end{equation}
	i
	\( \left\{ \varepsilon_{j,k} \st j,k \in \Z \right\} \)
	nezavisne jednako distribuirane varijable iz \( \mathrm N(0,1) \) definirane s
	\begin{equation}\label{eq:defepsjk}
		\varepsilon_{j,k} =
		\int_\R 2^{-j/2}e^{ik2^{-j}\xi}\wh \psi(-2^{-j}\xi)\D \wh B_\xi =
		\int_\R 2^{j/2} \psi(2^jt-k) \D B_t, \quad j,k\in \Z.
	\end{equation}
	Tada \( \left\{ B^H_t  \right\} \) ima reprezentaciju
	\begin{equation}\label{eq:valfbmrep}
		B^H_t = \sum_{j,k \in \Z}
		2^{-jH} \left( \Psi_H(2^jt - k) - \Psi_H(-k)  \right) \varepsilon_{j,k}, \quad t \in \R
	\end{equation}
	pri čemu konvergencija vrijedi i g.s.\ uniformno po \( t \) na svakom
	kompaktnom podskupu od \( \R \).
\end{teorem}

\begin{proof}
	Definirajmo prvo  \( \left\{ \varepsilon_{j,k }\right\} \) desnom stranom~\eqref{eq:defepsjk}, tj.\ \( \varepsilon_{j,k} = I(\psi_{j,k}) \).
	Po diskusiji iz odjeljka~\ref{sec:fbm}, znamo da su varijable \( \left\{ \varepsilon_{j,k} \right\} \)
	centrirane i normalno distribuirane. Varijancu dobijemo zbog izometričnosti
	Paley--Wienerovog integrala:
	\begin{equation}
		\E\varepsilon_{j,k}^2 = \E \left[ I(\psi_{j,k})^2 \right]  = \norm{\psi_{j,k}}^2_2 = 1.
	\end{equation}
	Na isti način možemo dobiti nekoreliranost: \( \E (\varepsilon_{j,k}\varepsilon_{j',k'}) = \skp{\psi_{j,k}}{\psi_{j',k'}} = 0 \) za \( (j,k) \neq (j',k') \). Ako varijable \( \left\{ \varepsilon_{j,k}  \right\} \)
	zajedno čine normalne slučajne vektore, tada je to dovoljno za nezavisnost.
	To je istina --- možemo nastaviti razmatranja iz odjeljka~\ref{sec:fbm} pa dokazati i ovo: ako su \( f_1, \ldots, f_n \in \L^2(\R) \)
	linearno nezavisne, tada je \( \left( I(f_1),\ldots,I(f_n) \right)  \) neprekidni normalni slučajni vektor.
	Naime, singularnost kovarijacijske matrice općenito povlači linearnu zavisnost varijabli, a to bi povuklo i
	linearnu zavisnost polaznih funkcija.

	Slijedeći pravila za modulacije i translacije (v.~komentar~\ref{kom:four-kovac})
	dobiva se da je Fourierova transformacija od \( \psi_{j,k} \) jednaka
	\( \xi \mapsto -2^{j/2}e^{ik2^{-j}\xi}\wh\psi(2^{-j}\xi) \).
	Konjugacijom, jer je Meyerov valić realan,
	dobiva se integrand u sredini~\eqref{eq:defepsjk}.
	Budući da je integral realan, konjugaciju dodamo bez promjene i time
	je~\eqref{eq:defepsjk} dokazano.

	Jer je \( \left\{ \psi_{j,k} \st j,k \in \Z \right\} \) ortonormirana baza
	prostora \( \L^2(\R) \) i Fourierova transformacija izometrija, familija
	\[
		\left\{ \xi \mapsto 2^{-j/2} e^{ik2^{-j}\xi}\wh\psi(-2^{-j}\xi) \st j,k \in \Z \right\}
	\]
	je također ortonormirana baza prostora \( \L^2(\R) \). Sada želimo dekomponirati
	funkciju \( \xi \mapsto \frac{e^{it\xi}-1}{(i\xi)^{H+1/2}} \), koja je očito u \( \L^2(\R) \), u toj bazi:
	\begin{equation} \label{eq:dekompfxi}
		\frac{e^{it\xi}-1}{(i\xi)^{H+1/2}} =
		\sum_{j,k \in \Z} c_{j,k}(t) 2^{-j/2}e^{ik2^{-j}\xi}\wh \psi(-2^{-j}\xi),
	\end{equation}
	gdje
	\begin{equation}
		\begin{aligned}
			c_{j,k}(t) & = 2^{-j/2} \int_\R \frac{e^{it\xi}-1}{(i\xi)^{H+1/2}}e^{-ik2^{-j}\xi}\wh \psi(2^{-j}\xi) \D \xi
			\\ &= 2^{-jH} \left( \Psi_H(2^jt-k)-\Psi_H(-k) \right),
		\end{aligned}
	\end{equation}
	a zadnja jednakost dobiva se supstitucijom \( \xi \leftarrow 2^{-j}\xi \).
	Integriranjem~\eqref{eq:dekompfxi} po \( \xi \), zamjenom sume i integrala (dominirana konvergencija zbog \( \Psi_H, \widehat{\psi} \in \mathcal S(\R) \)) i izometričnosti Paley--Wienerovog integrala
	dobiva se~\eqref{eq:valfbmrep}.

	Time smo dokazali da konvergencija u~\eqref{eq:valfbmrep} vrijedi u \( \L^2(\Omega) \), ali ne i g.s.\ uniformno po \( t \) na kompaktu. Dokaz te tvrdnje oslanja
	se na dva teorema izvan opsega ovog rada i može se naći u~\cite[tm.~3.15]{ayache}.

\end{proof}

Slijedi zanimljiva lema koja nam
omogućava da cijelu familiju slučajnih varijabli
ocijenimo jednom zajedničkom. Iz te perspektive
ju možemo zvati slučajnom konstantom.
Dokaz je adaptiran iz~\cite[str~459.]{jourfa-ayache}.
Primijetimo da tražimo da su varijable jednako distribuirane,
ali ne i da su nezavisne.

\begin{lema}\label{holpo-lema1}
	Neka je \( \left\{ \varepsilon_{j,k} \right\} \) familija jednako distribuiranih
	\( \mathrm N(0,1) \) varijabli. Tada postoji \( \Omega_1^* \) s \( \P(\Omega_1^*) = 1 \)
	takav da
	\begin{equation}
		\abs{\varepsilon_{j,k}(\omega)} \le C_1(\omega) \sqrt{\log(2 + \abs j + \abs k)}
		,\quad \omega \in \Omega_1^*, \ j,k \in \Z.
	\end{equation}
	gdje je \( C_1  \) slučajna varijabla čiji je svaki moment konačan.
\end{lema}

\begin{proof}
	Gledajmo prvo \( \N \)-indeksiranu familiju \( \left\{ \varepsilon_n \right\} \)
	i dokažimo analognu tvrdnju
	\begin{equation}\label{eq:holpo-lema1-anal}
		\abs{\varepsilon_n(\omega)} \le C(\omega) \sqrt{\log(2+n)}.
	\end{equation}
	Koristeći ocjenu iz dokaza teorema~\ref{tm:aditgauss} je:
	\begin{equation}
		\sum_{n=1}^\infty \P\left(\abs{\varepsilon_n} > a\sqrt{\log(2+n)}\right) \lesssim
		\sum_{n=1}^\infty (2+n)^{-a^2/2},
	\end{equation}
	što konvergira ako odaberemo \( a > \sqrt 2 \). Po Borel--Cantellijevoj lemi
	\begin{equation}
		\P\left(\abs{\varepsilon_n} > a\sqrt{\log(2+n)} \text{\ za beskonačno mnogo } n \in \N \right) = 0
	\end{equation}
	pa za \( \Omega_1^* \) stavimo komplement tog događaja, tj.\ na \( \Omega_1^* \)
	je dobro definirano vrijeme zaustavljanja
	\begin{equation}
		T(\omega) = \min \left\{ n \in \N \st m \ge n \implies
		\abs{\varepsilon_m(\omega)} \le a \sqrt{\log(2+m)}  \right\}.
	\end{equation}
	Ideja je sada \( C \) odabrati tako da u~\eqref{eq:holpo-lema1-anal} pokrije
	onih konačno \( n \) za koje tvrdnja već ne vrijedi. Nije očito da takva varijabla
	ima sve momente konačne. Ipak, možemo gledati i veću slučajnu konstantu
	\begin{equation}
		\widetilde C(\omega) = \max\limits_{0 \le m \le T(\omega)} \abs{\varepsilon_m(\omega)},
		\quad \omega \in \Omega_1^*
	\end{equation}
	i dokazati da su joj svi momenti konačni kako slijedi
	\begin{align}
		\E \widetilde C^p & =
		\sum_{n=1}^\infty \E \left[ \max\limits_{0 \le m \le T} \abs{\varepsilon_m}^p 1_{\left\{ T=n \right\}} \right]                                                  \\
		                  & \le \E \left[ \abs{\varepsilon_1}^p 1_{\left\{T=1\right\}}  \right]
		+ \sum_{n=2}^\infty \sum_{m=1}^n \E\left[ \abs{\varepsilon_m}^p 1_{\left\{ \abs{\varepsilon_{n-1}} > a\sqrt{\log(n+1)} \right\} } \right]                       \\
		                  & \le \E \abs{\varepsilon_1}^p + \sum_{n=1}^\infty \sum_{m=1}^n
		\sigma_{2p}^{1/2} \P \left( \abs{\varepsilon_{n-1}} > a\sqrt{\log(n+1)} \right)^{1/2}                                                                           \\
		                  & \lesssim \E \abs{\varepsilon_1}^p + \sigma_{2p}^{1/2}\sum_{n=2}^\infty (n+1)^{-a^2/4+1} < \infty \quad \left(\text{za } a > 2\sqrt2\right).
	\end{align}
	Pritom, druga nejednakost se dobije \holder ovom nejednakosti, uz oznaku
	\( \sigma_{2p} = \E \abs{\varepsilon_m}^{2p} \) (ne ovisi o \( m \)).
	Za treću primjenimo istu ocjenu repa normalne distribucije kao prije i koristimo
	jednaku distribuiranost.

	Jedan način za prelazak na \( (\Z\times\Z) \)-indeksiranu familiju je sljedeći:
	uzmemo bijekciju \( f\colon \Z\times\Z \to \N \) takvu da
	\[
		n = f(j,k) \implies n \lesssim \abs j + \abs k.
	\]
	To svojstvo ima uobičajena bijekcija (v.~\cite[lema~2]{jourfa-ayache} za eksplicitnu definiciju).
	Traženo se dobije iz~\eqref{eq:holpo-lema1-anal} uz odgovarajuću izmjenu konstante.

\end{proof}

Sljedeća propozicija, koju ćemo navesti bez dokaza (oslanja se na lemu; v.~\cite[prop.~3.17]{ayache} za referencu), daje ocjenu FBM preko valićne reprezentacije. Koristit ćemo
ju samo da jednom opravdamo zamjenu poretka integracije.

\begin{propozicija}\label{holpo-prop1}
	U oznakama teorema~\ref{holpo-teorem1} i leme~\ref{holpo-lema1}, postoji
	slučajna varijabla \( C_2 \) sa svim momentima konačnima takva
	da za sve \( t \in \R \) i \( \omega \in \Omega_1^* \) vrijedi
	\begin{align}
		\abs{B^H_t(\omega)}
		 & \le \sum_{j,k \in \Z} 2^{-jH} \abs{\Psi_H(2^jt-k)-\Psi_H(-k)} \abs{\varepsilon_{j,k}(\omega)}
		\\ &\le C_2(\omega)(1+\abs t)^H\sqrt{\abs{\log \abs{\log (2 + \abs t)}}}.
	\end{align}
\end{propozicija}

%%%%%%%%%%%%%%%%
Nakon toga, dokazat ćemo još neke korisne tvrdnje o \( \Psi_{\pm H} \)
i zatim dokazati svojevrsnu inverziju formule~\eqref{eq:valfbmrep}.

\begin{lema}\label{holpo-lema2}
	\begin{enumerate}[label=(\roman*)]
		\item Vrijedi \label{holpo-lema2-itemi}
		      \begin{equation}
			      \int_\R \Psi_{-H}(t) \D t = \int_\R \Psi_H(t) \D t = 0.
		      \end{equation}
		\item Za sve parove \( (j,k) \) i \( (j',k') \) cijelih brojeva \label{holpo-lema2-itemii}
		      \begin{equation}\label{eq:intpsihpsi-h}
			      2^{(j'+j)/2}\int_\R \Psi_H(2^{j'}t-k')\Psi_{-H}(2^jt-k) \D t =
			      \begin{cases}
				      1, \  & (j,k)=(j',k') \\
				      0, \  & \text{inače}.
			      \end{cases}
		      \end{equation}
	\end{enumerate}
\end{lema}

\begin{proof}
	Za~\ref{holpo-lema2-itemi} imamo
	\[
		\int_\R \Psi_{-H}(t) \D t = \sqrt{2\pi} \wh \Psi _{-H} (0) = 0,
	\]
	zbog \( \wh \Psi_{-H}(\xi) = (-i\xi)^{H+1/2}\wh \psi(\xi) \) (v.~komentar~\ref{kom:Psi}) i \( \wh \psi \)
	iščezava u okolini nule. Analogno se dokaže tvrdnja za \( \Psi_H \).

	Zatim~\ref{holpo-lema2-itemii} dobivamo iz
	\begin{align}\noindent
		 & 2^{( j'+j)/2} \int_\R \Psi_H(2^{j'}t-k')\Psi_{-H}(2^jt-k) \D t                                 \\
		 & = 2^{-(j'+j)/2}
		\int_\R  \scriptstyle{e^{-ik'2^{-j'}\xi}  (i2^{-j'}\xi)^{-H-1/2}\wh\psi(2^{-j'}\xi) \label{holpo-lema2-eq1}
		\ol{e^{-ik2^{-j}\xi}(-i2^{-j}\xi)^{H+1/2}\wh\psi(2^{-j}\xi)}\D \xi}                               \\
		 & = 2^{(H+1/2)(j'-j)-(j'+j)/2}\int_\R e^{-ik'2^{-j'}\xi}\wh\psi(2^{-j'}\xi)
		\ol{e^{-ik2^{-j}\xi}\wh\psi(2^{-j}\xi)} \D \xi                                                    \\
		 & = 2^{(H+1/2)(j'-j)+(j'+j)/2}\int_\R \psi(2^{j'}t-k')\psi(2^jt-k) \D t. \label{holpo-lema2-eq2}
	\end{align}
	Jednakost~\eqref{holpo-lema2-eq1} dobije se Parsevalovom jednakosti~\eqref{eq:parse}
	i rezultatima iz komentara~\ref{kom:four-kovac}. Primijenimo Parsevalovu jednakost u suprotnom smjeru
	da dobijemo~\eqref{holpo-lema2-eq2}. Na kraju dobivamo skalarni produkt dvaju valića i rezultat slijedi
	zbog ortonormiranosti familije.
\end{proof}

\begin{propozicija}\label{holpo-prop2}
	U oznakama teorema~\ref{holpo-teorem1} i leme~\ref{holpo-lema2}, vrijedi
	\begin{equation}
		\varepsilon_{j,k}(\omega) = 2^{j(1+H)}\int_\R B^H_t(\omega)\Psi_{-H}(2^jt-k) \D t, \quad \omega \in \Omega_1^*.
	\end{equation}
	%(kod~\cite{ayache} je \( B^H_t \) def.\ za \( t \in \R \), je li svejedno? \ref{??})
\end{propozicija}


\newcommand{\grn}{_\R}
\begin{proof}
	%%% odavde nadalje B^H_t je za t \in \R ; popraviti i drugdje
	Računamo:
	\begin{align}
		 & 2^{j(1+H)}\int\grn B^H_t\Psi_{-H}(2^jt-k) \D t                                                                                                                                                     \\
		 & = 2^{j(1+H)}\int\grn  \scriptstyle \left[ \sum\limits_{j',k'\in\Z} 2^{-j'H}\varepsilon_{j',k'}\left( \Psi_H(2^{j'}t-k')-\Psi_H(-k') \right) \right] \Psi_{-H}(2^jt-k) \D t \label{holpo-prop2-eq1} \\
		 & = \sum_{j',k'\in\Z} 2^{j+(j-j')H}\varepsilon_{j',k'}
		\int\grn \left( \Psi_H(2^{j'}t-k')-\Psi_H(-k) \right)\Psi_{-H}(2^jt-k) \D t                                                                                                                           \\
		 & = \sum_{j',k'\in\Z} 2^{j+(j-j')H}\varepsilon_{j',k'}
		\int\grn \Psi_H(2^{j'}t-k')\Psi_{-H}(2^jt-k) \D t \label{holpo-prop2-eq2}                                                                                                                             \\
		 & = \varepsilon_{j,k} \quad \left(\text{za } \omega \in \Omega_1^*\right). \label{holpo-prop2-eq3}
	\end{align}
	Zamjenu sume i integrala u~\eqref{holpo-prop2-eq1} možemo opravdati teoremom o
	dominiranoj konvergenciji --- prvo propozicijom~\ref{holpo-prop1}
	(zbog koje se zadržavamo na \( \omega \in \Omega_1^* \)) i zbog \( \Psi_{-H} \in \mathcal S(\R) \).
	Daljnje jednakosti dobiju se pomoću leme~\ref{holpo-lema2}, naime
	\begin{equation}
		\int\grn \Psi_H(-k)\Psi_{-H}(2^jt-k) \D t = 0, \quad j \neq 0,
	\end{equation}
	jer je riječ o~\eqref{eq:intpsihpsi-h} za \( j'=0 \), daje~\eqref{holpo-prop2-eq2} i direktna primjena daje~\eqref{holpo-prop2-eq3}.
\end{proof}

Sada smo spremni krenuti prema glavnom teoremu.
Definirat ćemo novu varijablu kao maksimum
jednog podskupa familije \( \left\{ \varepsilon_{j,k} \right\} \).
Lema~\ref{holpo-lema4} daje jedan opći rezultat,
a lema~\ref{holpo-lema3} dat će kontradiktorni rezultat
uz pretpostavku \( \alpha_{t_0} > H \).

% treba reći nešto o tome kako omega* u teoremu ne ovisi o t_0
\begin{definicija}
	Za \( \varepsilon_{j,k} \) iz teorema~\ref{holpo-teorem1}, za svaki \( j \in \N \)
	i \( \ell \in \Z \) označimo
	\begin{equation}
		\nu_j^\ell = \max\left\{ \abs{\varepsilon_{j, j\ell+m}} \st 0 \le m \le j-1 \right\}.
	\end{equation}
\end{definicija}

\begin{lema}\label{holpo-lema3}
	Pretpostavimo da za neki \( \omega \in \Omega_1^* \) (iz leme~\ref{holpo-lema1}) i neki
	\( t_0 \in \R \) vrijedi \( \alpha_{t_0}(\omega_0) > H \). Tada
	\begin{equation}
		\limsup\limits_{j \ub} \nu_j^{\ell_j(t_0)}(\omega_0) = 0,
	\end{equation}
	za \( \ell_j(t_0) = \max\left\{ \ell \in \Z \st j\ell \le 2^jt_0 \right\} \).
\end{lema}

\begin{proof}
	Pretpostavka \( \alpha_{t_0}(\omega_0) > H \) znači da postoje konstante \( c_0 > 0 \) i \( \varepsilon_0 > 0 \)
	takve da
	\begin{equation}
		\abs{B^H_t(\omega_0)- B^H_{t_0}(\omega_0)} \le
		c_0 \abs{t-t_0}^{H+\varepsilon_0}.
	\end{equation}
	Ocijenimo \( \abs{\varepsilon_{j,k}} \) počevši kombinacijom leme~\ref{holpo-lema2} i propozicije~\ref{holpo-prop2}:
	\begin{align}
		\abs{\varepsilon_{j,k}(\omega_0)} & =
		2^{j(1+H)}\abs{\int\grn \left[ B^H_t(\omega_0)-B^H_{t_0}(\omega_0) \right] \Psi_{-H}(2^jt-k) \D t}                                          \\
		                                  & \le 2^{j(1+H)} \int\grn \abs{B^H_t(\omega_0)-B^H_{t_0}(\omega_0)} \abs{\Psi_{-H}(2^jt-k)} \D t          \\
		                                  & \le c_02^{j(1+H)} \int\grn \abs{t-t_0}^{H+\varepsilon_0}\abs{\Psi_{-H}(2^jt-k)} \D t                    \\
		                                  & = c_02^{jH}\int\grn \abs{2^{-j}s-(t_0-2^{-j}k)}^{H+\varepsilon_0}\abs{\Psi_{-H}(s)} \D s                \\
		                                  & \le c_1 2^{-j\varepsilon_0}\left( 1+\abs{2^jt_0-k}\right)^{H+\varepsilon_0} \label{eq:holpo-lema3-eq1},
	\end{align}
	gdje \( c_1 > 0 \) ne ovisi o \( j \) i \( k \). Pretpostavimo nadalje da je \( j \ge 1 \)
	i \( k \) oblika \( k=j\ell_j(t_0)+m \) za \( 0 \le m \le j-1 \). Ekvivalentno je definiciji
	\( \ell_j(t_0) \) da
	\begin{equation}
		0 \le 2^jt_0 - j\ell_j(t_0) < j,
	\end{equation}
	što daje
	\begin{equation}
		\left( 1+\abs{2^jt_0-j\ell_j(t_0)-m} \right)^{H+\varepsilon_0}
		\le (2j)^{H+\varepsilon_0}.
	\end{equation}
	S tim iz~\eqref{eq:holpo-lema3-eq1} slijedi
	\begin{equation}
		\nu_j^{\ell_j(t_0)}(\omega_0) =
		\max \left\{ \abs{\varepsilon_{j,j\ell_j(t_0)+m}} \st 0 \le m \le j-1  \right\} \le
		c_1 2^{-j\varepsilon_0}(2j)^{H+\varepsilon_0}
	\end{equation}
	i \( \limsup\limits_{j\ub} \nu_j^{\ell_j(t_0)}(\omega_0)=0 \).
\end{proof}

\begin{lema}\label{holpo-lema4}
	Postoji \( \Omega_2^* \in \mathcal F \) takav da \( \P(\Omega_2^*)=1 \) i da za
	sve \( p \in \Z \) vrijedi
	\begin{equation}
		\liminf\limits_{j \ub} \min \left\{ \nu_j^\ell(\omega) \st
		(p-1)2^j \le j\ell \le (p+1)2^j \right\} \ge \frac 12,
		\quad \omega \in \Omega_2^*.
	\end{equation}
\end{lema}

\begin{proof}
	Definirajmo
	\begin{equation}
		B_j = \min\left\{ \nu_j^\ell \st (p-1)2^j \le j\ell \le (p+1)2^j \right\}
		\quad \text i \quad A_j = \left\{ B_j < \frac 12 \right\}, \quad j \ge 1.
	\end{equation}
	Fiksirajmo \( p \in \Z \) i, koristeći da su
	\( \varepsilon_{j,k} \) n.j.d.\ varijable jedinične normalne distribucije,
	za svaki \( j \ge 1 \) dobivamo
	\begin{align}
		\P(A_j) & \le \sum_{(p-1)2^j \le j\ell \le (p+1)2^j} \P\left(\bigcap_{m=0}^{j-1} \left\{ \abs{\varepsilon_{j, j\ell+m}} < \frac 12 \right\}\right) \\
		        & \le \left[ \frac {(p+1)2^j}j - \frac{(p-1)2^j}j + 1  \right]
		\left( \frac 1{\sqrt{2\pi}} \int_{-1/2}^{1/2}e^{-x^2/2} \D x \right)^j
		\lesssim j^{-1}\left( \sqrt{\frac 2\pi} \right)^j,
	\end{align}
	gdje zadnju ocjenu dobivamo ocjenjujući odozgo gustoću \( \mathrm N(0,1) \) konstantom.
	Slijedi
	\begin{equation}
		\sum_{j=1}^\infty \P(A_j) < \infty \quad \text i \quad
		\P\left( B_j < \frac 12 \text{ za beskonačno mnogo } j \right) = 0
	\end{equation}
	po Borel--Cantellijevoj lemi. Dobili smo traženo po definiciji limesa
	inferiora.
\end{proof}


Došli smo do glavnog teorema koji lagano
slijedi iz prošlih lema.

\begin{teorem}\label{holpo-tm}
	Postoji \( \Omega^* \in \mathcal F \) s \( \P(\Omega^*)=1 \)
	takav da za kritični točkovni \holder ov eksponent frakcionalnog Brownovog gibanja zadovoljava
	\begin{equation}
		\alpha_{t_0}(\omega) = H, \qquad \forall t_0 \in \R, \ \forall \omega \in \Omega^*.
	\end{equation}
\end{teorem}

\begin{proof}
	Stavimo \( \Omega^* = \Omega_1^* \cap \Omega_2^* \). Ako postoje \( \omega_0 \in \Omega^* \)
	i \( t_0 \in \R \) takvi da \( \alpha_{t_0}(\omega_0) > H \), lako se pokaže
	da je zaključak iz leme~\ref{holpo-lema3} u kontradikciji
	s definicijom događaja \( \Omega_2^* \).
\end{proof}

\begin{komentar}
	Primijetimo da naš izbor događaja \( \Omega^* \)
	ne ovisi o \( t_0 \). Ovako, tvrdnja \( \alpha_{t_0}(\omega) = H \) vrijedi
	istovremeno za sve \( t_0 \in \R \) dok god \( \omega \in \Omega^* \).
	U suprotnom, tvrdnja bi se donekle trivijalizirala (v.~\cite[str.~18.]{ayache}).
	Naime, ako za gaussovski proces \( \left\{ X_t \st t \in \R  \right\} \)
	s g.s.\ neprekidnim trajektorijama, neke \( t_0 \in \R \) i \( \delta> 0 \) vrijedi
	\begin{equation}\label{eq:zadnja}
		\limsup\limits_{h \downarrow 0} \frac{\E \abs{X_{t_0+h}-X_{t_0}}^2}{\abs h^{2\delta+\varepsilon}} = +\infty,
		\quad \forall \varepsilon > 0,
	\end{equation}
	tada postoji događaj \( \Omega_{t_0}^* \) s \( \P(\Omega_{t_0}^*) = 1 \)
	takav da \( \alpha_{t_0}(\omega) \le \delta \) za sve \( \omega \in \Omega_{t_0}^* \).
	Tvrdnja se dokaže gotovo jednako kao teorem~\ref{holunif-prop>}.
	Jasno,~\eqref{eq:zadnja} za \( \delta=H \) slijedi iz~\eqref{eq:deffbmalt}.

\end{komentar}

\end{document}
