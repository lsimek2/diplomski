\documentclass[main.tex]{subfiles}

\begin{document}
\nocite{*}

\section{Uniformna \holder -regularnost na kompaktnom skupu}\label{sec:holunif}
Do kraja rada bavit ćemo se naslovnom \holder -regularnosti
trajektorija FBM. Već smo neformalno najavili da kritični parametar \holder -regularnosti
odgovara parametru samosličnosti --- tvrdnju ćemo precizirati u
korolaru~\ref{holunif-tm}, teoremu~\ref{holunif-prop>} i teoremu~\ref{holpo-tm}. Naime,
razlikujemo dvije varijante \holder ovog uvjeta. Prvo, uniformno
kao u definiciji~\ref{def:holderf} ili u danoj točki
(v.\ definiciju~\ref{def:holpo}). \holder ov uvjet u točki ne mora nužno
vrijediti na okolini točke.

Za temu uniformne \holder -regularnosti neće nam trebati
valićne metode. Zato ćemo u ovom poglavlju rješenje prezentirati
ukratko i bez dokazivanja svih rezultata, a puno više pažnje
posvetiti sljedećem odjeljku~\ref{sec:holpo} u kojemu
ćemo konačno koristiti valićnu metodu za dokaz glavnog rezultata sadržanog
u teoremu~\ref{holpo-tm}.

Posvetimo se onda u ovom odjeljku pitanju uniformne \holder -regularnosti.
Prvo bez dokaza navodimo ključni teorem (v.~\cite[tm.~3.1]{se}; poznat kao Kolmogorov--Čencovljev teorem o neprekidnosti). Do kraja
rada podrazumijevamo fiksni \( H \in \left\langle 0,1 \right\rangle \).

\begin{teorem}\label{holunif-setm}
	Ako je \( \left\{ X_t \right\} \) slučajni proces takav da
	postoje konstante \( K, p, \beta > 0 \) takve da
	\begin{equation}
		\E \abs{X_t-X_s}^p \le K \abs{t-s}^{1+\beta}, \quad t,s \ge 0,
	\end{equation}
	tada ima modifikaciju \( \bigl\{ \widetilde X_t \bigr\} \)
	takvu da za svaki \( \alpha \in \left\langle 0, \beta/p \right\rangle \)
	i \( T > 0 \) vrijedi da je \( \bigl\{ \widetilde X_t \bigr\} \)
	\holder-regularan s parametrom \( \alpha \) na segmentu \( \left[ 0,T \right] \), tj.\
	\begin{equation}\label{eq:kolmcen}
		\sup\limits_{0 \le s, t \le T}
		\frac{\abs{\widetilde X_t-\widetilde X_s}}{\abs{t-s}^\alpha} < \infty.
	\end{equation}
\end{teorem}

\begin{korolar}\label{holunif-tm}
	Za svaki \( \alpha \in \left\langle0,H\right\rangle  \) FBM \( \left\{ B^H_t \right\} \) ima modifikaciju
	čije su trajektorije \holder -regularne s parametrom \( \alpha \) na svakom segmentu
	\( \left[ 0,T \right] \).
\end{korolar}

\begin{proof}
	Jer je \( B^H_t-B^H_s \) normalna varijabla, vrijedi ekvivalencija svih
	apsolutnih momenata (v.~\cite[str.~16]{ayache} za referencu) tj.\
	uz~\eqref{eq:deffbmalt} vrijedi da
	za sve \( p > 2 \) postoji konstanta \( K_p \) ovisna samo o \( p \) takva da
	\begin{equation}
		\E \abs{B^H_t-B^H_s}^p = K_p \left[ E \abs{B^H_t-B^H_s}^2 \right] ^{p/2} = K_p \abs{t-s}^{pH}.
	\end{equation}
	Po teoremu~\ref{holunif-setm} dobivamo tvrdnju korolara
	za sve \( \alpha \in \left\langle 0,H-1/p  \right\rangle  \). Kako je \( p \)
	proizvoljan, slijedi i tvrdnja za sve \( \alpha \in \left\langle 0,H \right\rangle \).
\end{proof}

Dokazali smo da \holder -regularnost vrijedi za sve parametre \( \alpha < H \),
ali ne i da ne vrijedi za \( \alpha > H \). To jamči sljedeći teorem
kojim smo riješili temu ovog odjeljka.

\begin{teorem}\label{holunif-prop>}
	Trajektorije FBM ne zadovoljavaju \holder ov uvjet ni za koji \( \alpha > H \).
\end{teorem}

\begin{proof}
	Pretpostavimo da postoji kompaktni skup \( I =[0,T] \)
	i \( \varepsilon > 0 \) takav da
	\begin{equation}
		\P\left( \sup\limits_{t,s \in I, t\neq s} \frac{\abs{B^H_t-B^H_s}}{\abs{t-s}^{H+\varepsilon}} < \infty \right) > 0.
	\end{equation}
	Sada se pozivamo na zakon nula-jedan za gaussovske procese, rezultat koji nam govori da se razna svojstva
	trajektorija gaussovskog procesa događaju s vjerojatnosti \( 0 \) ili \( 1 \) (v.~\cite[prop.~3.7]{ayache} za referencu).
	Posebice je \holder -regularnost jedno od tih svojstava, pa
	\begin{equation}
		\P \left( \sup\limits_{t,s \in I, t\neq s} \frac{\abs{B^H_t-B^H_s}}{\abs{t-s}^{H+\varepsilon}} < \infty  \right) = 1.
	\end{equation}
	Po Borellovoj\footnote{Christer Borell, a ne \'Emile Borel}(v.~\cite[prop.~3.7]{ayache} za referencu) nejednakosti slijedi
	\begin{equation}
		\E \left( \sup_{t,s \in I, t\neq s} \frac{\abs{B^H_t-B^H_s}^2}{\abs{t-s}^{2H+2\varepsilon}} \right) < \infty,
	\end{equation}
	no tu dolazimo do kontradikcije zbog~\eqref{eq:deffbmalt}:
	\begin{equation}
		\lim_{s \rightarrow t} \frac{\E \abs{B^H_t-B^H_s}^2}{\abs{t-s}^{2H+2\varepsilon}} = \lim_{s \rightarrow t} \abs{t-s}^{-2\varepsilon} = +\infty.
	\end{equation}
\end{proof}

\begin{komentar}
	Moguće je koristiti (v.~\cite[lema~3.6]{ayache}) jaču verziju teorema~\ref{holunif-setm}
	gdje je \holder ov uvjet~\eqref{eq:kolmcen} zamijenjen sa:
	\begin{equation}
		\abs{\wt X_t - \wt X_s} \le C \abs{t-s}^\alpha, \quad 0 \le t,s \le T
	\end{equation}
	gdje je \( C \) slučajna varijabla sa svim konačnim momentima.
	U~\eqref{eq:kolmcen} tvrdi se samo da je takva varijabla \( C \)
	dobro definirana.
	Naravno, u oba slučaja je nenegativna i ovisi samo o \( T \) i \( \alpha \).
	I u korolaru~\ref{holunif-tm} je onda moguće
	podrazumijevati tu jaču varijantu \holder ovog uvjeta.
\end{komentar}


\end{document}
