\documentclass[main.tex]{subfiles}

\begin{document}
\nocite{*}

\chapter{Valići}\label{ch-ha}
\section{Fourierova transformacija}\label{sec:ha-four}
Fourierova transformacija prvi je i temeljni pojam u harmonijskoj analizi.
Iako se pojavljuje u više varijanti i konteksta, sve ih povezuje
ideja da se funkcija prikaže kao zbroj (superpozicija) komponenti s određenim simetričnim svojstvima.
Tu se najčešće podrazumijevaju tzv.\ frekvencije
\( x \mapsto e^{ikx} \) gdje \( k \in \Z \) ili \( k \in \R \).
Možemo reći da na taj način dobivamo drugi pogled na funkciju, a njime
mnoge probleme puno lakše rješavamo.

Pojedinosti Fourierove transformacije ovise o funkcijskom prostoru
na kojem se definira. Pogledajmo prvo \( \L^2(\mathbb S^1) \),
pri čemu \( \mathbb S^1 \) možemo poistovjetiti
s torusom \( \mathbb T = \R/\Z \). Oboje se može interpretirati
i kao prostor periodičkih funkcija na \( \R \) s odgovarajućim svojstvom.
U~\cite[\textsection 6.2]{gogic} pokazuje se (pomoću Stone--Weierstrassovog i Luzinovog teorema) da je, uz normalizaciju mjere,
\( \left\{ x \mapsto e^{ikx} \st k \in \Z \right\} \)
ortonormirana baza (Hilbertovog prostora) \( \L^2(\mathbb S^1) \).
Uz oznake \( e_k(x) = e^{ikx} \) je stoga za sve \( f \in \L^2(\mathbb S^1) \)
\begin{equation}
	f = \sum_{k \in \Z} \wh f_k e_k,
	\qquad \wh f_k = \skp f{e_k} = \int_0^{2\pi} f(e^{i\theta}) e^{-ikx} \D \theta
\end{equation}
i lako se pokaže da je Fourierova transformacija \( \mathcal F \),
definirana s \( \mathcal F(f) = (\wh f_k)_{k \in \Z} \) za sve \( f \in \L^2(\mathbb S^1) \), izometrički izomorfizam
Hilbertovih prostora \( \L^2(\mathbb S^1) \) i \( \ell^2(\Z) \).
Primijetimo kako je u ovom slučaju dovoljno prebrojivo mnogo frekvencija za doći do
proizovljne \( f \), umjesto svih iz \( \R \). To nije slučaj na \( \L^2(\R) \) kojim ćemo se baviti u ovom radu.
Štoviše, uopće definirati Fourierovu transformaciju na \( \L^2(\R) \) pokazuje se izazovnijim.

Generalizaciju Fourierove transformacije nalazimo na lokalno kompaktnim Hausdorffovim Abelovim grupama (v.~\cite[\textsection 5.4,5.2]{gogic} za detalje).
Neka je \( G \) takva grupa s Haarovom mjerom \( \mu \).
Definiramo prvo \emph{karakter} na \( G \) kao neprekidni homomorfizam \( \chi \colon G \to \left( \mathbb S^1, \cdot \right) \).
Skup karaktera označavamo \( \wh G \) i zovemo \emph{Pontrjaginov dual} od \( G \). Za
\( f \in \L^1(G) \) definiramo Fourierovu transformaciju \( \wh f \colon \wh G \to \C \) sa
\begin{equation}\label{eq:fourgrupa}
	\wh f \left( \chi \right) =
	\int_G f(x) \ol{\chi(x)} \D \mu(x), \quad \chi \in \wh G.
\end{equation}
Najrelevantnije je ovdje da su za \( G = \R \) karakteri upravo sve
funkcije \( x \mapsto e^{i\xi x} \) za \( \xi \in \R \).
Posebice, \( \R \) (kao aditivna grupa) je izomorfan
svom Pontrjaginovom dualu. Uz njihovo poistovjećenje,
Fourierova transformacija dana s~\eqref{eq:fourgrupa}
jednaka je onoj iz definicije~\ref{def:four}.
Simetrična svojstva Fourierove transformacije, koja ćemo vidjeti u nastavku, upravo
proizlaze iz njihovog homomorfnog svojstva kao karaktera.

Još svakako vrijedi spomenuti i Fourierovu transformaciju na vjerojatnosnom prostoru.
Često se karakteristična funkcija zove Fourierovom transformacijom odgovarajuće
vjerojatnosne mjere odn.\ gustoće ako postoji.
To je u principu točno, iako uz konvenciju
\( \varphi_X(t)=\E(e^{itX}) \) kao npr.\ u~\cite[\textsection 13.1]{sarapa}
i \( \wh X \) danu analogonom~\eqref{eq:four} vrijedi
\begin{equation}
	\varphi_X(t) = \sqrt{2\pi}\wh X(-t), \quad t \in \R.
\end{equation}
Jasno, ova razlika neće ništa bitno promijeniti pa će tako
mnoga svojstva Fourierove trans\-formacije funkcija na \( \R \)
biti zajednička karakterističnim funkcijama. Primijetimo
kako je, zbog konačnosti mjere, karakteristična funkcija
dobro definirana bez obzira na integrabilnost slučajne
varijable \( X \). Situacija na prostorima s Lebesgueovom mjerom
bit će kompliciranija. Sada ćemo definirati
Fourierovu transformaciju na \( \L^1(\R) \) i pokazati
kako se može proširiti na pogodniji (Hilbertov)
prostor \( \L^2(\R) \).

\begin{definicija}\label{def:four}
	\emph{Fourierova transformacija} \( \mathcal F \colon \L^1(\R) \to \mathrm C_0(\R) \)
	definirana je s
	\begin{equation}\label{eq:four}
		\mathcal F(f)(\xi) = \wh f(\xi) =
		\frac 1{\sqrt{2\pi}} \int_\R f(x)e^{-i\xi x} \D \lambda(x), \quad
		\xi \in \R.
	\end{equation}
	za svaku \( f \in \L^1(\R) \). Funkcija \( \wh f \)
	zove se \emph{Fourierov transformat} ili također \emph{Fourierova trans\-for\-ma\-ci\-ja}.
\end{definicija}

\begin{komentar}\label{kom:four1}
	Moguće su varijacije na~\eqref{eq:four}.
	Primijetimo prvo konstantu \( 1/\sqrt{2\pi} \) --- njena poanta je da čuva
	\( L^2 \)-normu (jednom kad definiramo Fourierovu transformaciju na \( \L^2(\R) \)).
	Ako želimo uljepšati~\eqref{eq:four} moguće je Lebesgueovu mjeru
	zamijeniti s \( \mu = \lambda/\sqrt{2\pi} \) što je konvencija
	u~\cite[\textsection 9]{rudin}. To ćemo u ovom odjeljku napraviti i mi,
	s tim da, jer najčešće nije bitno, nećemo naglašavati da radimo
	na prostoru \( \L^1\left( \R, \mathcal B, \mu \right) \) umjesto
	\( \L^1\left(\R, \mathcal B, \lambda \right) \). Ipak, radi jasnoće naznačit ćemo
	mjeru \( \mu \) u integralima.

	Druge moguće varijacije su promjena predznaka u eksponentu od
	\( e^{-i\xi x} \) kao kod karakterističnih funkcija, ili izlučiti faktor
	\( 2\pi \) iz \( \xi \) tj.\ staviti \( \xi = 2\pi\eta \).
\end{komentar}

\begin{komentar}\label{kom:four2}
	% dobro def za L^1
	% u C_0
	Kod jednakosti oblika~\eqref{eq:four} prvo pitanje
	je je li uopće \( \wh f \) dobro definirana tj.\ koje su moguće domene
	za \( \mathcal F \). Drugo, koja je slika ili prikladan izbor kodomene?

	Najprije, \( \wh f \) je dobro definirana jer iz~\eqref{eq:four}
	zbog \( \abs{e^{-i\xi x}} = 1 \) odmah slijedi
	\( \abs{\wh f(\xi)} \le \norm f_1 \) za sve \( \xi \in \R \),
	a iz toga i \( \norm {\wh f}_\infty \le \norm f_1 \).
	Prethodno ujedno znači da je \( \mathcal F \) neprekidni linearni operator,
	ali možemo odabrati puno bolju kodomenu od \( \L^\infty(\R) \).

	Dokažimo da je \( \wh f \) neprekidna za sve \( f \in \L^1(\R) \).
	Za \( \xi, \eta \in \R \) vrijedi
	\begin{equation}
		\abs{\wh f(\xi) - \wh f(\eta)}
		\le \int_\R \abs{f(x)}\abs{e^{-i\xi x} - e^{-i\eta x}} \D \mu(x).
	\end{equation}
	Iz \( \abs{e^{-i\xi x} - e^{-i\eta x}} \un \) za \( \eta \rightarrow \xi \)
	po Lebesgueovom teoremu o dominiranoj konvergenciji (\( \le 2 \))
	slijedi i \( \abs{\wh f(\xi) - \wh f(\eta)} \rightarrow 0 \).
	Ovdje zapravo LTDK ne primjenjujemo na neprebrojivi skup funkcija,
	već koristimo Heineovu karakterizaciju kojom se limes funkcije u točki
	svodi na limes niza. Ovo pozadinsko značenje podrazumijevat ćemo i ubuduće.

	Dokažimo još da \( \wh f \) iščezava u beskonačnosti. Iz~\eqref{eq:four}
	odmah se dobije i
	\begin{equation}\label{eq:four2}
		\wh f (\xi) = -\int_\R f(x) e^{-i\xi(x + \pi/\xi)} \D \mu(x)
		= -\int_\R f(x-\pi/\xi) e^{-i\xi x} \D \mu(x), \quad \xi \in \R
	\end{equation}
	gdje se zadnje dobije supstitucijom \( x \leftarrow x - \pi/\xi \). Sada
	miješanjem~\eqref{eq:four} i~\eqref{eq:four2}
	je
	\begin{equation}
		\wh f(\xi) = \frac 12 \int_\R \left[ f(x)-f\left( x-\frac \pi \xi \right) \right] e^{-i\xi x} \D \mu(x), \quad \xi \in \R,
	\end{equation}
	to jest
	\begin{equation}
		\abs{\wh f (\xi)} \le \frac 12 \norm{f - f_{\pi/\xi}}_1, \quad \xi \in \R,
	\end{equation}
	gdje smo kao u teoremu~\ref{tm:trans} označili
	\( f_{y} = f(\cdot - y) \). Kad u tom teoremu dokažemo
	da je operator \( y \mapsto f_y \) neprekidan, odmah slijedi
	\( \lim\limits_{\xi \rightarrow \pm \infty} \abs{\wh f (\xi)} = 0 \)
	i \( \wh f \in \mathrm C_0(\R) \). Izbor kodomene
	\( \mathrm C_0(\R) \) je na neki način i najbolji. Iako
	\( \mathcal F \) nije surjekcija, slika joj je gusta u
	\( \mathrm C_0(\R) \) (v.~\cite[exercise~9.2]{rudin}).
\end{komentar}

Sljedeća propozicija daje neka osnovna svojstva Fourierove transformacije.

\begin{propozicija}\label{prop:form}
	Neka je \( f \in \L^1(\R) \) i neka su
	\( \alpha \) i \( \lambda > 0\) realni brojevi. Tada vrijedi:
	\begin{enumerate}[label=(\roman*)]
		\item Ako \( g(x)=f(x)e^{i\alpha x} \), tada \( \wh g(\xi) = \wh f(\xi-\alpha) \).\label{i:propform:a}
		\item Ako \( g(x)=f(x-\alpha) \), tada \( \wh g(\xi) = \wh f(\xi)e^{-i \alpha \xi} \).\label{i:propform:b}
		\item Ako \( g \in \L^1(\R) \) i \( h = f*g \), tada \( \wh h(\xi) = \wh f(\xi) \wh g(\xi) \).\label{i:propform:c}
		\item Ako \( g(x) = \ol{f(-x)} \), tada \( \wh g(\xi) = \ol{\wh f(\xi)} \).\label{i:propform:d}
		\item Ako \( g(x) = f(x/\lambda) \), tada \( \wh g(\xi) = \lambda \wh f(\lambda\xi) \).\label{i:propform:e}
		\item Ako \( g(x) = -ixf(x) \) i \( g \in \L^1(\R) \),
		      tada je \( \wh f \) diferencijabilna i \( \wh f'(\xi) = \wh g(\xi) \).\label{i:propform:f}
	\end{enumerate}
\end{propozicija}

\begin{proof}
	Točke~\ref{i:propform:a},~\ref{i:propform:b},~\ref{i:propform:d},~\ref{i:propform:e}
	trivijalno slijede iz definicije. Primijetimo kako~\ref{i:propform:a} i~\ref{i:propform:b}
	pokazuju dualnost između translacije i modulacije (množenja karakterom).
	Točka~\ref{i:propform:c} isto lako slijedi raspisom. Potrebna primjena Fubinijevog teorema
	valjana je zbog \( h \in \L^1(\R) \). Naime, opet Fubinijevim teoremom ali za
	nenegativne funkcije, lako se dobije \( \norm h_1 \le \norm f_1 \norm g_1 \).

	Dokažimo~\ref{i:propform:f}. Za različite \( \xi, \eta \in \R \) vrijedi
	\begin{equation}
		\frac{\wh f(\xi)-\wh f(\eta)}{\xi-\eta} =
		\int_\R f(x) e^{-i\xi x} u(x, \eta-\xi) \D \mu(x),
	\end{equation}
	gdje
	\begin{equation}
		u(x, \alpha) = \frac{1-e^{-i\alpha x}}{\alpha}.
	\end{equation}
	Iz ocjene \( \abs{e^{i\theta}-1} \le \abs \theta \) za \( \theta \in \R \)
	(v.~\cite[str.~514.]{sarapa}) je \( \abs{u(x, \alpha)} \le \abs x \) za
	\( \alpha \neq 0 \) i
	\( u(x, \alpha) \rightarrow -ix \) kad \( \alpha \un \) vrijedi
	Taylorovim razvojem. Stoga, po LTDK za \( \eta \rightarrow \xi \) je
	\begin{equation}
		\wh f'(\xi) = -i \int_\R xf(x) e^{-i \xi x} \D \mu (x), \quad \xi \in \R.
	\end{equation}
\end{proof}

\begin{komentar}
	\newcommand{\fourderi}[1]{\wh{{#1}^{\prime\phantom{.}}  }}
	Nakon točke~\ref{i:propform:f} možemo se pitati postoji li dualna tvrdnja
	tj.\ što je Fourierova transformacija derivacije. Rješenje je dosta
	jednostavno i to čini Fourierovu transformaciju korisnu u proučavanju
	diferencijalnih jednadžbi. Ako \( f,f' \in \L^1(\R) \) i ako je \( f \in \mathrm C^1(\R )\),
	tada se parcijalnom integracijom jednostavno dobije
	\begin{equation}\label{eq:fourderi}
		\fourderi f (\xi) = (i\xi)\wh f(\xi), \quad \xi \in \R.
	\end{equation}
	Induktivno, za veće stupnjeve glatkoće \( \ell \in \N \)
	je \( \wh{f^{(\ell)}}(\xi) = (i\xi)^\ell \wh f(\xi)  \).

	S druge strane, ako je \( f \) samo diferencijabilna g.s., identitet~\eqref{eq:fourderi}
	ne mora vrijediti (v.~\cite[exercise~9.6]{rudin}). Uzmemo li \( f=1_{[-1,1]} \)
	je \( f'=0 \) g.s.\ i \( \fourderi f = 0 \), ali
	\begin{equation}
		(i\xi)\wh f(\xi) = i\sqrt{\frac 2\pi} \sin \xi, \quad \xi \in \R.
	\end{equation}
\end{komentar}

Slijedi teorem u kojem uvodimo
operator translacije i dokazujemo njegovu neprekidnost.
Primijetimo razliku u odnosu na operatore \( \mathrm T_. \)
iz~\ref{??} --- ovdje je riječ o operatoru \( \R \to \L^p(\R) \),
a tamo \( \L^2(\R) \to \L^2(\R) \).

\begin{teorem}\label{tm:trans}
	Za funkciju \( f \) realne varijable i \( y \in \R \) definiramo
	translat \( f_y \) sa
	\begin{equation}
		f_y(x) = f(x-y), \quad x \in \R.
	\end{equation}
	Za  \( 1 \le p < \infty \) je preslikavanje \( y \mapsto f_y \)
	uniformno neprekidni operator između \( \R \) i \( \L^p(\R) \).
\end{teorem}

\begin{proof}
	Fiksirajmo \( p \) i \( \varepsilon > 0 \). Jer je \( \mrC_\mrc(\R) \) (neprekidne funkcije s kompaktnim nosačem)
	gust\footnote{tvrdnju nije teško dokazati, ali može se naći i kao theorem 3.14 u~\cite{rudin}} u \( \L^p(\R) \), postoji \( g \in \mrC_\mrc(\R) \) sa \( \supp g \subseteq \left[ -A, A \right] \)
	za neki \( A \in \R \) i \( \norm{f-g}_p < \varepsilon \). Funkcija \( g \)
	je uniformno neprekidna (neprekidna funkcija na segmentu) pa postoji \( \delta \in \left\langle0,A\right\rangle \) takav da
	\begin{equation}
		\abs{s-t} < \delta \implies \abs{g(s)-g(t)} < \left( 3A \right)^{-1/p} \varepsilon.
	\end{equation}
	Iz \( \abs{s-t} < \delta \) onda slijedi i
	\begin{equation}
		\int_\R \abs{g(x-s)-g(x-t)}^p \D x < \left( 3A \right)^{-1}\varepsilon^p
		\left( 2A + \delta \right) < \varepsilon^p,
	\end{equation}
	tj.\ \( \norm{g_s-g_t}_p < \varepsilon \). Sada je ključna invarijantnost
	Lebesgueove mjere na translaciju za
	\begin{equation}
		\begin{aligned}
			\norm{f_s-f_t}_p & \le \norm{f_s-g_s}_p + \norm{g_s-g_t}_p + \norm{g_t-f_t}_p                                         \\
			                 & = \norm{\left( f-g \right)_s}_p + \norm{g_s-g_t}_p + \norm{\left( g-f \right)_t}_p < 3\varepsilon,
			\quad \abs{s-t}<\delta.
		\end{aligned}
	\end{equation}
\end{proof}

Sada ćemo definirati dvije pomoćne funkcije. Definiramo \( H \) s
\begin{equation}
	H(\xi) = e^{-\abs \xi}, \quad \xi \in \R
\end{equation}
i za svaki \( \lambda > 0 \) definiramo
\begin{equation}
	h_\lambda(x) = \int_\R H(\lambda \xi) e^{i\xi x} \D \mu(\xi), \quad x \in \R,
\end{equation}
tako da
\begin{equation} \label{eq:hseintu1}
	h_\lambda(x) = \sqrt{\frac 2\pi} \frac \lambda{\lambda^2 + x^2}
	\quad \mathrm i \quad
	\int_\R h_\lambda(x) \D \mu(x) = 1.
\end{equation}
Možemo prepoznati \( H(\lambda \xi) \) i
\( h_\lambda \) kao redom karakterističnu funkciju i gustoću (u odnosu na mjeru \( \mu \))
centrirane Cauchyjeve distribucije s parametrom \( \lambda \). Dokazat ćemo tri
leme vezane za ove funkcije, da bi nam na kraju bile od koristi
u dokazivanju dva glavna teorema ovog odjeljka~---~teorema
inverzije~\ref{tm:inverzije} i Plancherelovog teorema~\ref{tm:planche}.
%Vidjet ćemo da su baš ovako odabrane funkcije od koristi jer...~\ref{??}.

Prvu lemu nećemo dokazivati jer slijedi direktnim računom
i već viđenom primjenom Fubinijevog teorema.

\begin{lema}\label{lema:four1}
	Ako je \( f \in \L^1(\R) \), tada je
	\begin{equation}
		(f*h_\lambda)(x) = \int_\R H(\lambda \xi)\wh f(\xi) e^{i\xi x} \D \mu(\xi), \quad x \in \R.
	\end{equation}
\end{lema}

\begin{lema}\label{lema:four2}
	Ako je \( g \in \L^\infty(\R) \) i ako je \( g \) neprekidna u nekoj točki \( x \in \R \),
	tada je
	\begin{equation}
		\lim\limits_{\lambda \un} (g * h_\lambda)(x) = g(x).
	\end{equation}
\end{lema}

\begin{proof}
	Ključno je da
	\begin{equation}
		h_\lambda(y) = \lambda^{-1} h_1\left( \frac y\lambda \right),
	\end{equation}
	što je vrlo bliska tvrdnja t.~\ref{i:propform:e} iz prop.~\ref{prop:form}.
	Druga bitna tvrdnja je da zbog~\eqref{eq:hseintu1} vrijedi
	\begin{equation}
		g(x) = \int_\R g(x) h_\lambda(y) \D \mu(y)
	\end{equation}
	čime dobivamo prvu jednakost u
	\begin{align}
		(g*h_\lambda)(x) - g(x) & = \int_\R \left[ g(x-y) - g(x) \right] h_\lambda(y) \D \mu(y)                                     \\
		                        & = \int_\R \left[ g(x-y) - g(x) \right] \lambda^{-1} h_1 \left( \frac y \lambda \right) \D \mu (y) \\
		                        & = \int_\R \left[ g(x-\lambda y) - g(x) \right] h_1(y) \D \mu(y). \label{eq:99z}
	\end{align}
	Pritom se~\eqref{eq:99z} dobije supstitucijom \( y \leftarrow \lambda y \).
	Integrand u~\eqref{eq:99z} je dominiran integrabilnom funkcijom
	\( 2\norm g_\infty h_\lambda (y) \) i konvergira u \( 0 \) kad \( \lambda \un \)
	za svaki \( y \in \R \) pa tvrdnja slijedi preko LTDK.
\end{proof}

\begin{lema}\label{lema:four3}
	Ako je \( 1 \le p < \infty \) i \( f \in \L^p(\R) \), tada
	\begin{equation}
		\lim\limits_{\lambda \un} \norm{f*h_\lambda - f}_p = 0.
	\end{equation}
\end{lema}

\begin{proof}
	Jer je \( h \in \L^q(\R) \) za \( q \) konjugirani eksponent od \( p \) je
	\( f * h_\lambda \in \L^1(\R) \) po Youngovoj nejednakosti. Analogno kao u lemi~\ref{lema:four2}
	počinjemo s
	\begin{equation}
		(f*h_\lambda)(x) - f(x) = \int_\R \left[ f(x-y)-f(x) \right] h_\lambda(y) \D \mu (y).
	\end{equation}
	Pomoću Jensenove jednakosti za konveksnu funkciju \( x \mapsto \abs x ^p \) slijedi
	\begin{equation}
		\abs{(f*h_\lambda)(x)-f(x)}^p \le \int_\R \abs{f(x-y)-f(x)}^p h_\lambda(y) \D\mu(y).
	\end{equation}
	Integriranjem po \( x \) i primjenom Fubinijevog teorema je i
	\begin{equation}\label{eq:four31}
		\norm{f*h_\lambda - f}_p^p \le \int_\R \norm{f_y-f}_p^p h_\lambda(y) \D\mu(y).
	\end{equation}
	Označimo \( g(y) = \norm{f_{-y}-f}_p^p \). Funkcija \( g \) je onda
	omeđena s \( 2^p\norm f_p^p \) i neprekidna po teoremu~\ref{tm:trans}
	te \( g(0)=0 \). Prepoznamo li desnu stranu u~\eqref{eq:four31} kao
	konvoluciju \( (g*h_\lambda)(0) \), možemo primijeniti lemu~\ref{lema:four2}
	da zaključimo da desna strana teži u \( 0 \) za \( \lambda \rightarrow 0 \).
\end{proof}

Spremni smo dokazati teorem inverzije koji pokazuje
kako se iz \( \wh f \) vratiti u \( f \).

\begin{teorem}\label{tm:inverzije}
	Neka je \( f \in \L^1(\R) \) i \( \wh f \in \L^1(\R) \). Ako
	\begin{equation}
		g(x) = \int_\R \wh f(\xi) e^{i \xi x} \D \mu(\xi), \quad x \in \R,
	\end{equation}
	tada je \( g \in \mrC_0(\R) \) i \( f=g \) g.s.
\end{teorem}

\begin{proof}
	Po lemi~\ref{lema:four1} je
	\begin{equation}
		(f * h_\lambda)(x) = \int_\R H(\lambda \xi) \wh f (\xi) e^{i \xi x} \D \mu(\xi).
	\end{equation}
	Integrand zdesna je omeđen s \( \abs{\wh f(\xi)} \) za svaki \( \xi \in \R \)
	pa uz \( H(\lambda \xi) \rightarrow 1 \) za \( \lambda \un \) zaključujemo da desna
	strana konvergira prema \( g(x) \) za svaki \( x \in \R \) po LTDK. Iz
	leme~\ref{lema:four3} i činjenice da konvergencija po normi
	povlači konvergenciju g.s.\ nekog podniza, zaključujemo
	\begin{equation}
		\lim\limits_{n \ub} (f * h_{\lambda_n})(x) = f(x) \ \mathrm{g.s.},
	\end{equation}
	za neki niz \( (\lambda_n)_n \) s \( \lambda_n \un \). Time
	slijedi  \( f(x) = g(x) \) g.s.

	Tvrdnja da je \( g \in \mrC_0(\R) \) dokaže se analogno
	istom za \( \wh f \) u komentaru~\ref{kom:four2}.
\end{proof}

Posljedica teorema inverzije je i rezultat koji se ponekad sam
naziva \emph{teorem jedinstvenosti}: ako je
\( \wh f = \wh g \), tada je i \( f=g \) g.s.

Teorem inverzije pokazuje još jedno lijepo svojstvo Fourierove transformacije,
a to je da je inverz vrlo sličan njoj samoj. Zapravo, da bismo mogli
govoriti o pravom inverzu \( \mathcal F^{-1} \) potrebno je za
domenu od \( \mathcal F \) uzeti prostor na kojemu
je \( \mathcal F \) automorfizam.
To ćemo postići Plancherelovim teoremom koji omogućuje da \( \mathcal F \)
proširimo na \( \L^2(\R) \), štoviše pokazujući da je riječ o izometričkom
izomorfizmu \( \L^2(\R) \to \L^2(\R) \).

Osnovna ideja je prvo se restringirati na \( \L^1(\R) \cap \L^2(\R) \)
što je gust potprostor od \( \L^2(\R) \) (\( \mrC_\mrc(\R) \) gust je u svakom
\( \L^p(\R) \), \( 1 \le p < \infty \)). Tada svakako postoji
jedinstveno neprekidno proširenje na cijeli \( \L^2(\R) \) (v.~\cite[prop.~1.6.8~(iv)]{gogic}),
no i dalje je pitanje kakva će svojstva imati to proširenje.

\begin{teorem}\label{tM:planche}
	Postoji izometrički izomorfizam \( \mathcal F \colon \L^2(\R) \to \L^2(\R) \)
	takav da, uz oznaku \( \wh f = \mathcal F(f) \), vrijedi:
	\begin{enumerate}[label=(\roman*)]
		\item Ako je \( f \in \L^1(\R) \cap \L^2(\R) \), onda je \( \wh f \)
		      dana s~\eqref{eq:four}.
		\item Ako je
		      \begin{equation}
			      \varphi_A(\xi) = \int_{-A}^A f(x)e^{-i\xi x} \D \mu(x)
			      \quad \mathrm i \quad
			      \psi_A(x) = \int_{-A}^A \wh f(\xi) e^{i\xi x} \D \mu(\xi),
		      \end{equation}
		      onda \( \norm{\varphi_A - \wh f}_2 \un \) i \( \norm{\psi_A - f}_2 \un \)
		      kad \( A \ub \).
	\end{enumerate}
\end{teorem}

\begin{proof}
	Definirajmo \( \mathcal F \) na gustom potprostoru \( \L^1(\R) \cap \L^2(\R) \) s~\eqref{eq:four}
	i proširimo do operatora na \( \L^2(\R) \). Da dokažemo da je riječ o izometričkom izomorfizmu
	dovoljno je sljedeće:
	\begin{itemize}
		\item Vrijedi
		      \begin{equation}\label{eq:fl2izo}
			      \norm {\wh f}_2 = \norm f_2, \quad f\in \L^1(\R) \cap \L^2(\R).
		      \end{equation}
		      Po neprekidnosti norme, iz~\eqref{eq:fl2izo} tvrdnja
		      odmah slijedi za sve \( f \in \L^2(\R) \).

		\item Slika od \( \mathcal F \) gusta je u \( \L^2(\R) \).
		      To je dovoljno da \( \mathcal F \) bude surjekcija. Neka je
		      \( f \in \L^2(\R) \) proizvoljna i \( (f_n)_n \) niz u \( \im \mathcal F \)
		      takav da \( f_n \rightarrow f \) i \( \mathcal F g_n = f_n \)
		      za neku \( g_n \in \L^2(\R) \) za svaki \( n \in \N \).
		      Zbog izometričnosti je niz \( (g_n)_n \) također Cauchyjev,
		      pa zbog potpunosti prostora \( \L^2(\R) \) postoji limes
		      \( g = \lim\limits_{n \ub} g_n \) te zbog neprekidnosti
		      od \( \mathcal F \) je \( \mathcal Fg = f \).
		      %Dakle, \( f \in \im \mathcal F \).
	\end{itemize}

	Dokažimo prvo~\eqref{eq:fl2izo}.
\end{proof}

\end{document}
