\documentclass[main.tex]{subfiles}

\begin{document}
\nocite{*}

\chapter{Valići}\label{ch-ha}
\section{Fourierova transformacija}\label{sec:ha-four}
Fourierova transformacija prvi je i temeljni pojam u harmonijskoj analizi.
Iako se pojavljuje u više varijanti i konteksta, sve ih povezuje
ideja da se funkcija prikaže kao zbroj (superpozicija) komponenti s određenim simetričnim svojstvima.
Tu se najčešće podrazumijevaju tzv.\ frekvencije
\( x \mapsto e^{ikx} \) gdje \( k \in \Z \) ili \( k \in \R \).
Možemo reći da na taj način dobivamo drugi pogled na funkciju, a njime
mnoge probleme puno lakše rješavamo.

Pojedinosti Fourierove transformacije ovise o funkcijskom prostoru
na kojem se definira. Pogledajmo prvo \( \L^2(\mathbb S^1) \),
pri čemu \( \mathbb S^1 \) možemo poistovjetiti
s torusom \( \mathbb T = \R/\Z \). Oboje se može interpretirati
i kao prostor periodičkih funkcija na \( \R \) s odgovarajućim svojstvom.
U~\cite[\textsection 6.2]{gogic} pokazuje se (pomoću Stone--Weierstrassovog i Luzinovog teorema) da je, uz normalizaciju mjere,
\( \left\{ x \mapsto e^{ikx} \st k \in \Z \right\} \)
ortonormirana baza (Hilbertovog prostora) \( \L^2(\mathbb S^1) \).
Uz oznake \( e_k(x) = e^{ikx} \) je stoga za sve \( f \in \L^2(\mathbb S^1) \)
\begin{equation}
	f = \sum_{k \in \Z} \wh f_k e_k,
	\qquad \wh f_k = \skp f{e_k} = \int_0^{2\pi} f(e^{i\theta}) e^{-ikx} \D \theta
\end{equation}
i lako se pokaže da je Fourierova transformacija \( \mathcal F \),
definirana s \( \mathcal F(f) = (\wh f_k)_{k \in \Z} \) za sve \( f \in \L^2(\mathbb S^1) \), izometrički izomorfizam
Hilbertovih prostora \( \L^2(\mathbb S^1) \) i \( \ell^2(\Z) \).
Primijetimo kako je u ovom slučaju dovoljno prebrojivo mnogo frekvencija za doći do
proizovljne \( f \), umjesto svih iz \( \R \). To nije slučaj na \( \L^2(\R) \) kojim ćemo se baviti u ovom radu.
Štoviše, uopće definirati Fourierovu transformaciju na \( \L^2(\R) \) pokazuje se izazovnijim.

Generalizaciju Fourierove transformacije nalazimo na lokalno kompaktnim Hausdorffovim Abelovim grupama (v.~\cite[\textsection 5.4,5.2]{gogic} za detalje).
Neka je \( G \) takva grupa s Haarovom mjerom \( \mu \).
Definiramo prvo \emph{karakter} na \( G \) kao neprekidni homomorfizam \( \chi \colon G \to \left( \mathbb S^1, \cdot \right) \).
Skup karaktera označavamo \( \wh G \) i zovemo \emph{Pontrjaginov dual} od \( G \). Za
\( f \in \L^1(G) \) definiramo Fourierovu transformaciju \( \wh f \colon \wh G \to \C \) sa
\begin{equation}\label{eq:fourgrupa}
	\wh f \left( \chi \right) =
	\int_G f(x) \ol{\chi(x)} \D \mu, \quad \chi \in \wh G.
\end{equation}
Najrelevantnije je ovdje da su za \( G = \R \) karakteri upravo sve
funkcije \( x \mapsto e^{i\xi x} \) za \( \xi \in \R \).
Posebice, \( \R \) (kao aditivna grupa) je izomorfan
svom Pontrjaginovom dualu. Uz njihovo poistovjećenje,
Fourierova transformacija dana s~\eqref{eq:fourgrupa}
jednaka je onoj iz definicije~\ref{def:four}.
Simetrična svojstva Fourierove transformacije, koja ćemo vidjeti u nastavku, upravo
proizlaze iz njihovog homomorfnog svojstva kao karaktera.

Još svakako vrijedi spomenuti i Fourierovu transformaciju na vjerojatnosnom prostoru.
Često se karakteristična funkcija zove Fourierovom transformacijom odgovarajuće
vjerojatnosne mjere odn.\ gustoće ako postoji.
To je u principu točno, iako uz konvenciju
\( \varphi_X(t)=\E(e^{itX}) \) kao npr.\ u~\cite[\textsection 13.1]{sarapa}
i \( \wh X \) danu analogonom~\eqref{eq:four} vrijedi
\begin{equation}
	\varphi_X(t) = \sqrt{2\pi}\wh X(-t), \quad t \in \R.
\end{equation}
Jasno, ova razlika neće ništa bitno promijeniti pa će tako
mnoga svojstva Fourierove trans\-formacije funkcija na \( \R \)
biti zajednička karakterističnim funkcijama. Primijetimo
kako je, zbog konačnosti mjere, karakteristična funkcija
dobro definirana bez obzira na integrabilnost slučajne
varijable \( X \). Situacija na prostorima s Lebesgueovom mjerom
bit će kompliciranija. Sada ćemo definirati
Fourierovu transformaciju na \( \L^1(\R) \) i pokazati
kako se može proširiti na pogodniji (Hilbertov)
prostor \( \L^2(\R) \).

\begin{definicija}\label{def:four}
	\emph{Fourierova transformacija} \( \mathcal F \colon \L^1(\R) \to \mathrm C_0(\R) \)
	definirana je s
	\begin{equation}\label{eq:four}
		\mathcal F(f)(\xi) = \wh f(\xi) =
		\frac 1{\sqrt{2\pi}} \int_\R f(x)e^{-i\xi x} \D \lambda.
	\end{equation}
	za svaku \( f \in \L^1(\R) \). Funkcija \( \wh f \)
	zove se \emph{Fourierov transformat} ili također \emph{Fourierova trans\-for\-ma\-ci\-ja}.
\end{definicija}

\begin{komentar}\label{kom:four1}
	Moguće su varijacije na~\eqref{eq:four}.

\end{komentar}

\begin{komentar}\label{kom:four2}
	% dobro def za L^1
	% u C_0
\end{komentar}




\end{document}
