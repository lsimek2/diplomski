\documentclass[main.tex]{subfiles}

\begin{document}
\section{Multirezolucijska analiza}\label{sec:mra}
Multirezolucijska analiza pojam je koji uopćuje
pojmove i rezultate iz komentara~\ref{kom:tmhaar2}
i dokaza teorema~\ref{tm:haar}.
Nije riječ samo o elegantnom misaonom modelu, nego
se pokazuje puno više~---~omogućava sistematičnu konstrukciju
ortonormiranih valićnih baza.

\setlist{leftmargin=*}
\begin{definicija}
	\emph{Multirezolucijsku analizu} (skraćeno MRA)
	čine niz \( (V_j)_{j \in \Z} \) zatvorenih potprostora od \( \L^2(\R) \)
	i \emph{skalirajuća funkcija} \( \phi \in V_0 \) takvi da vrijede sljedeći uvjeti.
	\begin{enumerate}[label=(MRA\arabic*)]
		\item Niz \( (V_j)_{j \in \Z} \) je padajući:
		      \begin{equation}
			      \cdots \subseteq V_2 \subseteq V_1 \subseteq V_0
			      \subseteq V_{-1} \subseteq V_{-2} \subseteq \cdots
		      \end{equation} \label{i:mra1}
		\item Vrijedi
		      \begin{equation}
			      \bigcap_{j \in \Z} V_j = \left\{ 0 \right\}
			      \quad \text i \quad
			      \ol{\bigcup_{j \in \Z} V_j} = \L^2(\R).
		      \end{equation} \label{i:mra2}

		\item Za sve \( j \in \Z \), \( f \in V_j \) ako i samo ako \( f(2^j\cdot) \in V_0 \). \label{i:mra3}
		\item Ako je \( f \in V_0 \), tada je \( f(\cdot - k) \in V_0 \) za sve \( k \in \Z \). \label{i:mra4}
		\item Familija\footnote{analogno kao za valiće, definiramo \( \phi_{j,k}(x)=2^{-j/2}\phi(2^{-j}x-k)\)} \( \left\{ \phi_{0,k} \st k \in \Z \right\} \) je ortonormirana baza od
		      \( \L^2(\R) \). \label{i:mra5}
	\end{enumerate}
\end{definicija}

Da je MRA moguće povezati s ortonormiranom valićnom bazom tvrdi sljedeći teorem.

\begin{teorem}\label{tm:mra}
	Ako niz potprostora \( (V_j)_{j \in \Z} \) i funkcija \( \phi \)
	čine multirezolucijsku analizu i označimo s \( P_j \)
	projektor na potprostor \( V_j \),
	tada postoji
	%(ne nužno jedinstvena)
	ortonormirana valićna baza \( \left\{ \psi_{j,k} \st j,k\in\Z \right\} \) na \( \L^2(\R) \) takva da
	\begin{equation}\label{eq:mrapj}
		P_{j-1} = P_j + \sum_{k \in \Z} \skp \cdot{\psi_{j,k}}\psi_{j,k}, \quad j \in \Z.
	\end{equation}
\end{teorem}
\noindent Cilj ovog odjeljka je dokazati teorem~\ref{tm:mra} i konstruirati
traženu valićnu bazu. Napomenimo kako se uvjet~\ref{i:mra5} može zamijeniti slabijim
gdje je \( \left\{ \phi_{0,k} \st k \in \Z \right\} \) samo Rieszova baza (v.~\cite[\textsection 5.3]{daub}).

Za svaki \( j \in \Z \) označimo s \( W_j \) ortogonalni komplement od \( V_j \)
u \( V_{j-1} \), tj.\ vrijedi
\begin{equation}
	V_{j-1} = V_j \oplus W_j, \quad j \in \Z.
\end{equation}
Slijedi \( W_j \perp W_{j'} \) za \( j \neq j' \). Naime, ako je bez smanjenja općenitosti
\( j > j' \), onda \( W_j \subseteq V_{j'} \perp W_{j'} \). Zato možemo pisati
\begin{equation}
	V_j = V_J \oplus \bigoplus_{k=1}^{J-j+1} W_{J-k}, \quad j < J.
\end{equation}
Skupa s uvjetom~\ref{i:mra2} to daje
\begin{equation}\label{eq:l2jopluswj}
	\L^2(\R) = \bigoplus_{j \in Z} W_j.
\end{equation}
Jednakost~\eqref{eq:mrapj} znači da je
\( \left\{ \psi_{j,k} \st k \in \Z \right\} \) ortonormirana baza
na \( W_j \) za svaki \( j \in \Z \). Zbog~\eqref{eq:l2jopluswj}
je to i dovoljno da \( \left\{ \psi_{j,k} \st j,k \in \Z \right\} \)
bude ortonormirana baza na \( \L^2(\R) \). Zbog uvjeta~\ref{i:mra3}
dovoljno je samo da je \( \left\{ \psi_{0,k} \st k \in \Z \right\} \)
ortonormirana baza od \( W_0 \).

Zbog~\ref{i:mra3} je \( \left\{ \psi_{-1, k} \st k \in \Z \right\} \)
ortnormirana baza od \( V_{-1} \). Kako je \( \phi \in V_0 \subseteq V_{-1} \)
je
\begin{equation}\label{eq:mraphih}
	\phi = \sum_{k \in \Z} h_k \phi_{-1,k}, \quad h_k = \skp \phi{\phi_{-1,k}},
	\quad \sum_{k \in \Z} \abs{h_k}^2 = 1.
\end{equation}
Identitet~\eqref{eq:mraphih} može se zapisati kao
\begin{equation}
	\phi(x) = \sqrt 2 \sum_{k \in \Z} h_k \phi(2x-k), \quad x \in \R.
\end{equation}
Primijenimo Fourierovu transformaciju na obje strane. Po rezultatima iz komentara~\ref{kom:four-kovac}
\newcommand{\mathgs}{\ \ \mathrm{g.s.}}
\begin{equation}
	\wh \phi(\xi) = \frac 1{\sqrt 2} \sum_{k \in \Z}
	h_k e^{-i\xi k/2} \wh \phi(\xi/2) \mathgs
\end{equation}
To možemo zapisati i kao
\begin{equation}\label{eq:mraphim0}
	\wh \phi(\xi) = m_0(\xi/2)\wh \phi(\xi/2) \mathgs, \quad
	m_0(\xi) = \frac 1{\sqrt 2} \sum_{k \in \Z} h_k e^{-i\xi k}.
\end{equation}
Očito je \( m_0 \) \( 2\pi \)-periodična funkcija te je
zbog~\eqref{eq:mraphih} \( m_0 \in \L^2([0,2\pi]) \).

Pogledajmo što daje ortonormiranost od \( \left\{ \phi_{0,k} \st k \in \Z \right\} \).
Vrijedi
\begin{align}
	\delta_{0,k} & = \int_\R \phi(x) \ol{\phi(x-k)} \D x
	\stackrell{\eqref{eq:parse}}{=} \int_\R \abs{\wh \phi(\xi)}^2 e^{i\xi k} \D \xi                                                         \\
	             & = \int_0^{2\pi} e^{i \xi k} \sum_{\ell \in \Z} \abs{\wh \phi \left( \xi + 2\pi\ell \right)}^2 \D \xi, \label{eq:mraal11}
\end{align}
gdje se do~\eqref{eq:mraal11} dolazi zbog periodičnosti \( \xi \mapsto e^{i\xi k} \).
Po teoremu~\ref{tm:inverzije} slijedi
\begin{equation}\label{eq:mrasumwhphi}
	\sum_{\ell \in \Z} \abs{\wh \phi\left( \xi + 2\pi\ell \right)}^2 = \frac 1 {2\pi} \mathgs
\end{equation}
Pomoću~\eqref{eq:mraphim0} i oznake \( \zeta = \xi/2 \) je
\begin{equation}\label{eq:mrasuma2}
	\sum_{\ell \in \Z} \abs{m_0\left( \zeta + \pi\ell \right)}^2
	\abs{\wh \phi \left( \zeta + \pi\ell \right)}^2 = \frac 1{2\pi} \mathgs
\end{equation}
Zbog periodičnosti \( m_0 \) možemo~\eqref{eq:mrasuma2} podijeliti
na dva dijela za parne i neparne \( \ell \) i koristeći opet~\eqref{eq:mrasumwhphi}
dobiti
\begin{equation}\label{eq:mram0sum1}
	\abs{m_0(\zeta)}^2 + \abs{m_0(\zeta+\pi)}^2 = 1 \mathgs
\end{equation}

Nadalje pogledajmo što dobivamo za \( f \in W_0 \), što je ekvivalentno
\( f \in V_{-1} \) i \( f \perp V_0 \).
Zbog prvog
\begin{equation}
	f = \sum_{k \in \Z} f_k \phi_{-1,k}, \quad
	f_k = \skp f{\phi_{-1, k}}.
\end{equation}
Opet primijenimo Fourierovu transformaciju da dobijemo
\begin{equation}\label{eq:mrawhf}
	\wh f(\xi) = \frac 1{\sqrt 2} \sum_{k \in \Z}
	f_k e^{-i\xi k/2} \wh \phi(\xi/2) = m_f(\xi/2)\wh\phi(\xi/2),
	\quad m_f(\xi) = \frac 1{\sqrt 2} \sum_{k \in \Z} f_k e^{-i\xi k/2} \mathgs
\end{equation}
Funkcija \( m_f \) opet je \( 2\pi \)-periodična i u \( \L^2([0,2\pi]) \).
Uvjet \( f \perp V_0 \) znači \( f \perp \phi_{0,k} \) za sve \( k \in \Z \) tj.
\begin{equation}
	0 = \int_\R f(x) \ol{\phi(x-k)} \D x \stackrell{\eqref{eq:parse}}=
	\int_\R \wh f(\xi) \ol{\wh \phi(\xi)} e^{i\xi k} \D \xi.
\end{equation}
Ponovo koristeći periodičnost \( \xi \mapsto e^{i\xi k} \) je
\begin{equation}
	\int_0^{2\pi} e^{i\xi k} \sum_{\ell \in \Z}
	\wh f\left( \xi + 2\pi\ell \right) \ol{\wh \phi \left( \xi + 2\pi\ell \right)}
	\D \xi = 0
\end{equation}
i po teoremu~\ref{tm:inverzije}
\begin{equation}\label{eq:mrafphi}
	\sum_{\ell \in \Z} \wh f(\xi + 2\pi\ell) \ol{\wh \phi(\xi + 2\pi\ell)} = 0 \mathgs
\end{equation}
Konvergencija u~\eqref{eq:mrafphi} je apsolutna
pa možemo permutirati elemente i primijeniti~\eqref{eq:mrasumwhphi}
da dobijemo
\begin{equation}\label{eq:mramfm0}
	m_f(\zeta)\ol{m_0(\zeta)} + m_f(\zeta+\pi) \ol{m_0(\zeta+\pi)} = 0 \mathgs
\end{equation}
Zbog~\eqref{eq:mram0sum1} ne mogu
\( m_0(\zeta) \) i \( m_0(\zeta + \pi) \) istovremeno biti \( 0 \)
(osim na skupu mjere \( 0 \)), pa~\eqref{eq:mramfm0}
povlači da postoji \( 2\pi \)-periodična funkcije \( \lambda \) takve da
\begin{equation}\label{eq:mramflambda}
	m_f(\zeta) = \lambda(\zeta) \ol{m_0(\zeta + \pi)}, \quad
	\lambda(\zeta) + \lambda(\zeta + \pi)=0 \mathgs
\end{equation}
Drugo možemo zapisati i kao
\begin{equation}\label{eq:mralambdanu}
	\lambda(\zeta) = e^{i\zeta} \nu(2\zeta) \mathgs
\end{equation}
za neku \( 2\pi \)-periodičnu \( \nu \).
Uvrštavanjem~\eqref{eq:mralambdanu} i~\eqref{eq:mramflambda}
u~\eqref{eq:mrawhf} dobiva se
\begin{equation}\label{eq:mradefwhf}
	\wh f(\xi) = e^{i\xi/2} \ol{m_0(\xi/2 + \pi)} \nu(\xi) \wh \phi(\xi/2) \mathgs
\end{equation}
za \( 2\pi \)-periodičnu \( \nu \).

Ovo nam daje ideju \( \psi \) definirati preko
\begin{equation}\label{eq:mradefwhpsi}
	\wh \psi(\xi) = e^{i\xi/2} \ol{m_0(\xi/2 + \pi)} \wh \phi(\xi/2).
\end{equation}
Formalno je
\begin{equation}
	\wh f(\xi) = \nu(\xi) \wh \psi(\xi) = \sum_{k \in \Z} \nu_k e^{-i\xi k} \wh \psi(\xi) \mathgs,
\end{equation}
tj.\ inverzijom
\begin{equation}
	f = \sum_{k \in \Z} \nu_k \psi_{0,k}.
\end{equation}
S \( \nu_k \) definiraju se Fourierovi koeficijenti od \( \nu \),
budući da je \( 2\pi \)-periodična.
Zato su preko~\eqref{eq:mradefwhpsi} dani dobri
kandidati za ortonormiranu bazu od \( W_0 \).

Preostaje dokazati da je \( \left\{ \psi_{0,k} \st k \in \Z \right\} \)
doista ortonormirana baza od \( W_0 \).
Po prethodnoj analizi slijedi da je uopće \( \psi \in W_0 \).
Dokažimo ortonormiranost familije.
Koristimo se već viđenim idejama. Najprije
\begin{align}
	\int_\R \psi(x) \ol{\psi(x-k)} \D x
	= \int_\R e^{i \xi k} \abs{\wh \psi(\xi)}^2 \D \xi
	= \int_0^{2\pi} e^{i\xi k} \sum_{\ell \in \Z} \abs{\wh \psi(\xi + 2\pi\ell)}^2 \D \xi,
\end{align}
pa zatim
\begin{align}
	\sum_{\ell \in \Z} \abs{\wh \psi(\xi + 2\pi\ell)}^2
	       & \stackrell{\eqref{eq:mradefwhpsi}}{=}
	\sum_{\ell \in \Z} \abs{m_0(\xi/2 + \pi\ell + \pi)}^2 \abs{\wh \phi(\xi/2 + \pi)}                \\
	       & = \abs{m_0(\xi/2 + \pi)}^2 \sum_{n \in \Z} \abs{\wh \phi(\xi/2 + 2\pi n)}^2
	\qquad & (\text{za neparne } \ell)                                                               \\
	       & \phantom{=} + \abs{m_0(\xi/2)}^2 \sum_{n \in \Z} \abs{\wh \phi(\xi/2 + \pi + 2\pi n)}^2
	\qquad & (\text{za parne } \ell)                                                                 \\
	       & \stackrell{\eqref{eq:mrasumwhphi}}= \frac 1{2\pi}
	\left[ \abs{m_0(\xi/2)}^2 + \abs{m_0(\xi/2 + \pi)}^2 \right] \mathgs                             \\
	       & \stackrell{\eqref{eq:mram0sum1}}=
	\frac 1{2\pi} \mathgs
\end{align}
Sada odmah slijedi \( \skp{\psi}{\psi_{0,k}} = \delta_{0,k} \).

Na kraju, dokažimo da \( \left\{ \psi_{0,k} \st k \in \Z \right\} \)
doista generiraju cijeli \( W_0 \). Dovoljno je da se svaka \( f \in W_0 \)
može zapisati kao
\begin{equation}
	f = \sum_{k \in \Z} \gamma_k \psi_{0,k}, \quad \sum_{k \in \Z} \abs{\gamma_k}^2 < \infty,
\end{equation}
ili, ekvivalentno
\begin{equation}
	\wh f(\xi) = \gamma(\xi) \wh \psi(\xi) \mathgs
\end{equation}
za neku \( 2\pi \)-periodičnu i kvadratno integrabilnu funkciju \( \gamma \).
U~\eqref{eq:mradefwhf} već smo dobili
\( \wh f(\xi) = \nu(\xi) \wh \psi(\xi) \) za \( \nu \) \( 2\pi \)-periodičnu,
pa je dovoljno još dokazati da je \( \nu \) kvadratno integrabilna.
Vrijedi
\begin{align}
	\int_0^{2\pi} \abs{\nu(\xi)}^2 \D \xi & \stackrell{\eqref{eq:mralambdanu}}= 2\int_0^\pi \abs{\lambda(\xi)}^2 \D \xi \\
	                                      & \stackrell {\eqref{eq:mram0sum1}}=
	\int_0^\pi \abs{\lambda(\xi)}^2 \left[ \abs{m_0(\xi + \pi)}^2 + \abs{m_0(\xi)}^2 \right] \D \xi                     \\
	                                      & \stackrell{\eqref{eq:mramflambda}}=
	\int_0^{2\pi} \abs{\lambda(\xi)}^2 \abs{m_0(\xi + \pi)}^2 \D \xi                                                    \\
	                                      & \stackrell{\eqref{eq:mramflambda}}=
	\int_0^{2\pi} \abs{m_f(\xi)}^2 \D \xi = \pi \sum_{k \in \Z} \abs{f_k}^2 = \pi \norm f_2^2 < \infty.
\end{align}
\end{document}
