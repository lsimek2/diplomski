\documentclass[main.tex]{subfiles}

\begin{document}
\nocite{*}

\bigskip
Sada možemo krenuti prema glavnom rezultatu odjeljka. Definirajmo najprije nekoliko novih pojmova.
Za \( x \in \R^d \) neka je \( K_\varepsilon(x) \) \( \varepsilon \)-okolina točke \( x \), tj.\footnote{zapis implicira da je  \( \abs{\, \cdot \,} \) norma na \( \R^d \) --- kako su
	sve norme ekvivalentne na konačnodimenzionalnom vektorskom prostoru, odabir nije bitan}\
\( K_x(\varepsilon) = \left\{ y \in \R^d \st \abs{y-x} < \varepsilon \right\} \).
Neka su \( \Ptrans_{s,t} \) tranzicijske funkcije na \( \R^d \). Definiramo
\begin{equation}\label{eq:defalpha}
	\alpha_{\varepsilon, T}(u) = \sup \bigl\{ \Ptrans_{s,t}(x, K^c_\varepsilon(x)) \st x \in \R^d, \ s, t \in [0, T], \ 0 \le t-s \le u \bigr\}.
\end{equation}
Nadalje neka je \( M \subseteq \left[ 0, \infty\right\rangle \). Za fiksni \( \omega \) kažemo da
\( X_t(\omega)  \) ima \( \varepsilon \)-\emph{oscilaciju} \( n \) \emph{puta u} \( M \) ako postoje
\( t_0 < \cdots < t_n \) gdje \( t_k \in M \) takvi da
\[
	\abs{X_{t_k}(\omega)-X_{t_{k-1}}(\omega)} > \varepsilon, \quad 1 \le k \le n.
\]
Ako tvrdnja vrijedi za sve \( n \in \N \), kažemo da ima \emph{beskonačno čestu} \( \varepsilon \)\emph{-oscilaciju}. Definiramo još
\begin{gather}
	\begin{aligned}
		\Omega_2 = \bigl\{ \omega \in \Omega \st & \lim_{s \in \Q, s \downarrow t} X_s(\omega) \mathrm{\ postoji \ u \ } \R^d \mathrm{\ za \ sve \ } t \ge 0 \mathrm{ \ i \ } \\                                & \lim_{s \in \Q, s \uparrow t} X_s(\omega) \mathrm{\ postoji \ u \ } \R^d \mathrm{\ za \ sve \ } t > 0	\bigr\},
	\end{aligned}
	\\
	A_{N, k} = \bigl\{ \omega \in \Omega \st X_t(\omega) \mathrm{\ nema \ beskona\check cno \ \check cestu \ } \frac 1k \mathrm{\text{-oscilaciju}\ u \ } [0, N] \cap \Q \bigr\},
	\\
	\Omega_2' = \bigcap_{N=1}^\infty \bigcap_{k=1}^\infty A_{N, k},
	\\
	B(p, \varepsilon, M) = \bigl\{  \omega \in \Omega \st X_t(\omega) \mathrm{\ ima \ } \varepsilon \mathrm{\text{-oscilaciju}\ } p \mathrm{\ puta \ u \ } M  \bigr\}.
\end{gather}
Pokaže se \( \Omega_2' \in \mathcal F \). Prije glavnog rezultata, potrebno nam je nekoliko lema.
\begin{lema}
	Vrijedi \( \Omega_2' \subseteq \Omega_2 \).
\end{lema}

\begin{proof}
	Neka je \( \omega \in \Omega_2' \) i pretpostavimo da ne postoji limes zdesna u \( t \ge 0 \) (analogno za limes slijeva). To znači da postoji niz \( (s_n)_n \)
	racionalnih brojeva takvih da \( s_n \downarrow t \) i da ne postoji limes \( \lim_n X_{s_n}(\omega) \). To znači da niz \( (X_{s_n}(\omega))_n \) nije Cauchyjev tj.\ postoji
	\( \varepsilon > 0 \) takav da
	\begin{equation} \label{lema112cauchy}
		(\forall n_0 \in \N)(\exists m,n \ge n_0 \mathrm{\ takvi \ da \ } \abs{X_{s_n}(\omega) - X_{s_m}(\omega)} > 2\varepsilon ).
	\end{equation}
	Naći ćemo beskonačno čestu \( \varepsilon \)-oscilaciju i time kontradikciju. Neka je \( s_1' = s_n \) proizvoljan. Tvrdimo da postoji
	\( s_2' = s_m \) gdje \( m > n \) takav da \( \abs{X_{s_1'}(\omega) - X_{s_2'}(\omega)} > \varepsilon \). U suprotnom bi bilo
	\begin{equation}
		\begin{aligned}
			\abs{X_{s_m}(\omega) - X_{s_k}(\omega)} & \le \abs{X_{s_m}(\omega) - X_{s_1'}(\omega)} + \abs{X_{s_1'}(\omega) - X_{s_k}(\omega)} \\
			                                        & \le \varepsilon + \varepsilon = 2\varepsilon,
		\end{aligned}
	\end{equation}
	za sve \( m, k > n \) što je u kontradikciji s~\eqref{lema112cauchy}. Iteriranjem ovog postupka dolazimo do
	niza \( (s_n')_n \) takvog da \( \abs{X_{s_k'}(\omega) - X_{s_{k+1}'}(\omega)} > \varepsilon \) za \( k \in \N \).
\end{proof}

\begin{lema}\label{markovlema2}
	Ako je \( \left\{ X_t \right\} \) stohastički neprekidan i \( \P(\Omega_2') = 1 \), tada ima modifikaciju \( \left\{ X_t' \right\} \)
	takvu da je \( t \mapsto X_t'(\omega) \) \cadlag \ za svaki \( \omega \in \Omega \).
\end{lema}

\begin{proof}
	Definiramo
	\begin{equation} X_t'(\omega) =
		\begin{cases}
			\lim\limits_{s \in \Q, s \downarrow t}  X_s(\omega), \quad & \omega \in \Omega_2',      \\
			0, \quad                                                   & \omega \not \in \Omega_2'.
		\end{cases}
	\end{equation}
	Dokažimo da je \( t \mapsto X_t'(\omega) \) neprekidna zdesna. Za \( \omega \not \in \Omega_2' \) je očito, pa neka
	\( \omega \in \Omega_2' \). Neka je \( t \ge 0 \) proizvoljan i \( (t_n)_n \) takav da \( t_n \downarrow t \). Za svaki \( n \in \N \)
	definiramo \( \delta_n > 0 \) takav da \( t_n + \delta_n \in \Q \) i takav da desna strana u
	\begin{equation}
		\abs{X'_t(\omega) - X'_{t_n}(\omega)} \le \abs{X'_t(\omega) - X_{t_n+\delta_n}(\omega)} + \abs{X_{t_n+\delta_n}(\omega) - X'_{t_n}(\omega)}
	\end{equation}
	teži k nuli kad \( n \ub \). Jasno je da je to moguće po definiciji skupa \( \Omega_2 \supseteq \Omega_2' \). Time smo dokazali neprekidnost zdesna,
	a postojanje limesa slijeva dokaže se sličnom tehnikom.

	Preostaje dokazati da je \( \left\{ X_t \right\} \) doista modifikacija polaznog procesa. Neka je \( t \ge 0 \) i \( (s_n)_n \) niz racionalnih brojeva
	takav da \( s_n \downarrow t \). Po stohastičkoj neprekidnosti je \( X_{s_n} \konvp X_t \), a po definiciji procesa \( \left\{ X_t' \right\} \) i \( \P(\Omega_2')=1 \) vrijedi
	\( X_{s_n} \konvgs X_t' \). Slijedi \( \P(X_t=X_t')=1 \).
	%\begin{equation}
	%	t_n + \delta_n \in \Q, \quad \abs{X_t'(\omega) - X_{t_n+\delta_n}} \le \frac 1{2n}, \quad \abs{X'_{t_n(\omega)} - X_{t_n+\delta_n}(\omega)} \le \frac 1{2n},
	%\end{equation}
\end{proof}

\begin{lema}\label{markovlema3}
	Neka je \( p \in \N \),
	\[
		0 \le s_1 < \cdots < s_m \le u \le t_1 < \cdots < t_n \le v \le T
	\]
	i \( M = \left\{ t_1, \ldots, t_n \right\} \). Ako je \( \left\{ X_t \right\} \) Markovljev proces
	s familijom tranzicijskih funkcija \( \left\{ \Ptrans_{s,t} \right\} \) i početnom vrijednosti \( a \), tada
	\begin{equation}
		\E^{0,a}[Z \cdot 1_{B(p, 4\varepsilon, M)}] \le \E^{0,a}[Z](2\alpha_{\varepsilon, T}(v-u))^p,
	\end{equation}
	za svaku \( Z = g(X_{s_1},\ldots,X_{s_m}) \), gdje je \( g \) Borelova i nenegativna.
\end{lema}

\begin{proof}
	Pretpostavimo bez smanjenja općenitosti da je \( \left\{ X_t \right\} \) sam trajektorijska reprezentacija. Tvrdnju dokazujemo indukcijom po \( p \).
	Neka je \( p=1 \). Definiramo događaje
	\begin{align}
		C_k & = \left\{ \abs{X_{t_j}-X_u} \le 2\varepsilon, \text{\ za \ }  1 \le j \le k-1  \text{\, i \,} \abs{X_{t_k} - X_u} > 2\varepsilon \right\}, \\
		D_k & = \left\{ \abs{X_v - X_{t_k}} > \varepsilon \right\}, \quad 1 \le k \le n.
	\end{align}
	Događaji \( C_k \) su disjunktni. Tvrdimo
	\begin{equation}
		\begin{aligned}
			B(1, 4\varepsilon, M) & \subseteq \bigcup_{k=1}^n \left\{ \abs{X_{t_k}-X_u} > 2\varepsilon  \right\} = \bigcup_{k=1}^n C_k \\
			                      & \subseteq \left\{ \abs{X_u-X_v} \ge \varepsilon  \right\} \cup \bigcup_{k=1}^n (C_k \cap D_k).
		\end{aligned}
	\end{equation}
	Prva inkluzija vrijedi jer ako \( \omega \in  B(1, 4\varepsilon, M) \) i \( 4\varepsilon < \abs{X_{t_j}(\omega)-X_{t_k}} \le \abs{X_{t_j}-X_u} + \abs{X_{t_k}-X_u} \)
	pa je barem jedan od sumanada zdesna \( > 2\varepsilon \). Ako \( \omega \in C_k \) iz \( 4\varepsilon \abs{X_{t_j}-X_{t_k}} \le \abs{X_{t_j}-X_u} + \abs{X_u-X_v} + \abs{X_{t_k}-X_v}  \)
	i \( \abs{X_u-X_v} < \varepsilon \) slijedi \( \abs{X_{t_k} - X_v} > \varepsilon \), stoga vrijedi druga inkluzija. Primjenom matematičkog očekivanja dobivamo
	\begin{equation}
		\begin{aligned}
			\E^{0,a}[Z \cdot 1_{B(1, 4\varepsilon, M)}] & \le \E^{0,a}\left[Z \cdot 1_{\left\{ \abs{X_u-X_v} \ge \varepsilon \right\}}\right] +
			\sum_{k=1}^n \E^{0,a}[Z 1_{C_k} 1_{D_k}]                                                                                                               \\
			                                            & \le \E^{0,a} [Z] \alpha_{\varepsilon, T}(v-u) + \sum_{k=1}^n \E^{0,a}[Z1C_k]\alpha_{\varepsilon, T}(v-u) \\
			                                            & \le 2 \E^{0,a} [Z] \alpha_{\varepsilon, T}(v-u),
		\end{aligned}
	\end{equation}
	pri čemu drugu nejednakost dobivamo primjenom Markovljevog svojstva~\eqref{markov-svojstvo}, nakon čega
	dobivenu varijablu možemo uniformno ograničiti konstantom \( \alpha_{\varepsilon, T}(v-u) \) po definiciji~\eqref{eq:defalpha}. Time je dokazana baza indukcije.

	Pretpostavimo da tvrdnja vrijedi za \( p-1 \) i dokažimo da vrijedi za \( p \). Definiramo događaje
	\begin{equation}
		\begin{aligned}
			F_k & = \left\{ X_t \text{\ ima\ } 4\varepsilon \text{-oscilaciju\ } p-1 \text{\ puta u} \left\{ t_1, \ldots, t_k \right\} \text{\ ali ne i u} \left\{ t_1, \ldots, t_{k-1} \right\}  \right\}, \\
			G_k & = \left\{ X_t \text{\ ima\ } 4\varepsilon \text{-oscilaciju jednom u} \left\{t_{k+1},\ldots,t_n\right\}\right\}, \quad 1 \le k \le n.
		\end{aligned}
	\end{equation}
	Događaji \( F_k \) su disjunktni i očito
	\begin{equation}
		B(p-1, 4\varepsilon, M) = \bigcup_{k=1}^n F_k \quad \mathrm i \quad B(p, 4\varepsilon, M) \subseteq \bigcup_{k=1}^n (F_k \cap G_k).
	\end{equation}
	Nastavimo kao u dokazu baze te primjenom pretpostavke i slučaja \( p=1 \) dobivamo traženo.
\end{proof}

\bigskip
Slijedi glavni rezultat odjeljka koji daje dovoljne uvjete da bi Markovljev proces imao modifikaciju s \cadlag \ odn.\ neprekidnim trajektorijama.

\begin{teorem}\label{markov-dov}
	Neka je \( \left\{ X_t \right\} \) Markovljev proces na \( \left( \Omega, \mathcal F, \P \right) \) s familijom tranzicijskih funkcija
	\( \left\{ \Ptrans_{s,t} \right\} \). Ako vrijedi
	\begin{equation} \label{markov-dov1}
		\lim\limits_{u \downarrow 0} \alpha_{\varepsilon, T}(u) = 0, \quad \forall \varepsilon >0, \ \forall T > 0,
	\end{equation}
	tada \( \left\{ X_t \right\} \) ima modifikaciju \( \left\{ X_t' \right\} \) takvu
	da je trajektorija \( t \mapsto X_t'(\omega) \) \cadlag \ za svaki \( \omega \in \Omega \).
	Ako vrijedi jači uvjet
	\begin{equation} \label{markov-dov2}
		\lim\limits_{u \downarrow 0} \frac 1u \alpha_{\varepsilon, T}(u) = 0, \quad \forall \varepsilon > 0,\  \forall T > 0,
	\end{equation}
	tada \( \left\{ X_t \right\} \) ima modifikaciju \( \left\{ X_t' \right\} \) takvu da postoji \( \Omega_1 \in \mathcal F \)
	takav da \( \P(\Omega_1) = 1 \) i \( \omega \in \Omega_1 \) povlači da je trajektorija \( t \mapsto X_t'(\omega) \) neprekidna.
\end{teorem}

\begin{proof}
	Pretpostavimo prvo~\eqref{markov-dov1}. Po lemi~\ref{markovlema2} dovoljno je dokazati
	\( \P(\Omega_2') = 1 \), a za to je opet dovoljno \( \P(A_{N,k}^c) = 0 \) za sve \( N, k \in N \).
	Fiksirajmo \( N \) i \( k \). Po~\eqref{markov-dov1} odaberimo \( \ell \)
	takav da \( 2\alpha_{1/(4k), N}(N/\ell) < 1 \). Definiramo \( t_j = \frac{Nj}\ell \) za \( 0 \le j \le \ell \) i \( I_j = \left[ t_{j-1}, t_j \right] \cap \Q \) za \( j > 0 \). Vrijedi
	\begin{align}
		\P(A_{N,k}^c) & = \P\left( X_t \text{\ ima\ } \frac 1k \text{-oscilaciju beskonačno često u\ } [0,N]\cap \Q \right)            \\
		              & \le \sum_{j=1}^\ell \P\left( X_t \text{\ ima\ } \frac 1k \text{-oscilaciju beskonačno često u\  } I_j \right)  \\
		              & =\sum_{j=1}^\ell \lim\limits_{p \ub} \P \left[ B\left( p, \frac 1k, I_j \right) \right]. \label{eq:markovtm11}
	\end{align}
	Uzmimo fiksni \( I_j \) i \( \left\{ s_1, \ldots, s_n \right\} \subset I_j \). Po lemi~\ref{markovlema3} je
	\begin{equation}
		\P\left[ B\left(p, 1/k, \left\{ s_1, \ldots, s_n  \right\}\right) \right] \le
		\left( 2\alpha_{\frac 1{4k}, N} \left( \frac N\ell \right)  \right)^p.
	\end{equation}
	Aproksimirajući \( I_j \) odozgo konačnim skupovima (\( n \ub \)) slijedi
	\begin{equation}
		\P\left[ B\left(p, 1/k, I_j\right) \right] \le
		\left( 2\alpha_{\frac 1{4k}, N} \left( \frac N\ell \right)  \right)^p,
	\end{equation}
	što znači da~\eqref{eq:markovtm11} teži u \( 0 \), a stoga i \( \P(A_{N,k}^c)=0 \) kako smo htjeli.

	Pretpostavimo sada~\eqref{markov-dov2}. Dovoljno je dokazati da za svaki \( N \in \N \)
	postoji \( H_N \in \mathcal F \) takav da \( \P(H_N)=1 \) i
	\[
		H_N \subseteq \left\{ X'_t = X'_{t-}, \ \forall t \in \left\langle 0,N \right] \right\}.
	\]
	gdje je \( X'_{t-} \) limes slijeva u \( t \). Neka je \( N \) fiksan i definiramo za \( \ell \in \N \) i \( \varepsilon > 0 \)
	\begin{align}
		R_{\ell, \varepsilon}   & = \sum_{j=1}^\ell 1_{\abs{X'_{t_j}-X'_{t_{j-1}}} > \varepsilon},                                            \\
		R_{\varepsilon}(\omega) & = \card \left\{ t \in \left\langle0, N\right] \st \abs{X'_t(\omega)-X'_{t-}(\omega)} > \varepsilon \right\}
	\end{align}
	pa je \( R_\varepsilon(\omega) \le \liminf\limits_{\ell \ub} R_{\ell, \varepsilon}(\omega) \). Vrijedi
	\begin{equation}
		\E(R_{\ell, \varepsilon}) =
		\sum_{j=1}^\ell \P\left(\abs{X'_{t_j} - X'_{t_{j-1}}} > \varepsilon\right) \le \ell \alpha_{\varepsilon, N}\left( \frac N\ell \right).
	\end{equation}
	Po~\eqref{markov-dov2} imamo \( \lim\limits_{\ell \ub} \E(R_{\ell, \varepsilon}) = 0 \) i po Fatouovoj lemi \( \E\left(\liminf\limits_{\ell \ub} R_{\ell, \varepsilon}\right) = 0 \). Stavimo
	\[
		H_N = \bigcap_{k=1}^\infty \left\{ \liminf\limits_{\ell \ub} R_{\ell, 1/k} = 0 \right\}.
	\]
	Zadovoljili smo željeni uvjet jer \( H_N \subseteq \left\{ R_\varepsilon = 0, \ \forall \varepsilon > 0 \right\} \).
\end{proof}
% Satov (11.4)?
\bigskip
Sada samo trebamo dokazati da \levy jev po distribuciji proces zadovoljava~\eqref{markov-dov1} da
bismo mogli pojačati teorem~\ref{bddlpd} i zaključiti da postoji \( 1 \)-\( 1 \)
korespondencija između beskonačno djeljivih distribucija i \levy jevih procesa. Štoviše, za aditivni
proces po distribuciji, koji je Markovljev s prostorno homogenim tranzicijskim funkcijama, imamo:
\[
	\Ptrans_{s, t}(x, K_\varepsilon^c(x)) = \Ptrans_{s, t}(0, K_\varepsilon^c(0)) = \P(\abs{X_s-X_t} > \varepsilon).
\]
Slijedi \( \alpha_{\varepsilon, T}(u) \un \) kad \( u \un \). Za taj zaključak nam je zapravo
potrebna i uniformnost stohastičke neprekidnosti (v.~\cite[lema~9.6]{sato}). Dakle, po
teoremu~\ref{markov-dov} svaki \levy jev po distribuciji (aditivan po distribuciji)
proces ima modifikaciju s \cadlag \ trajektorijama, tj.\ koja je \levy jev (aditivni) proces.

Na korak smo i do dokaza da Brownovo gibanje definirano preko definicije~\ref{def:brown} postoji, no fali još svojstvo~\ref{enum:browndef2} za što bismo trebali dokazati
i~\eqref{markov-dov2}. Pokazuje se svaki \levy jev gaussovski (tj.\ takav da
je svaka konačnodimenzionalna distribucija normalna) proces u \( \R \) g.s.\ ima neprekidne trajektorije,
što je tvrdnja sljedećeg teorema. Tvrdnja vrijedi i za aditivni proces u općem \( \R^d \) (v.~\cite[tm.~11.7]{sato}). Čak vrijedi i obrat, tj.\ jedini aditivni procesi s neprekidnim trajektorijama su gaussovski (v.~\cite[\textsection21]{sato}).

\begin{teorem}\label{tm:aditgauss}
	Ako je \( X_t \) \levy jev gaussovski proces na \( \R \), tada gotovo sigurno vrijedi da je trajektorija \( t \mapsto X_t(\omega) \) neprekidna, tj.\ postoji \( \Omega_1 \in \mathcal F \) takav da \( \P(\Omega_1)=1 \) i \( \omega\in \Omega_1\) povlači da je trajektorija \( t \mapsto X_t(\omega) \) neprekidna.
\end{teorem}

\begin{proof}
	Neka je bez smanjenja općenitosti \( X_1 \sim \mathrm N(0,1) \). Ocjenu
	\[
		\int_c^\infty e^{-\frac{x^2}2} \D x \le
		\int_c^\infty e^{-\frac{x^2}2} \left( 1 + \frac 1{x^2} \right) \D x
		= \frac 1c e^{-\frac{c^2}2}, \quad c > 0,
	\]
	gdje se jednakost dobije direktno antiderivacijom, iskoristimo za
	\begin{align}
		\P(\abs{X_s} > \varepsilon) & = \frac 2{\sqrt{2\pi s}} \int_\varepsilon^\infty e^{-\frac{-x^2}{2s}} \D x      \\
		                            & = \frac 2{\sqrt{2 \pi}} \int_{\varepsilon/\sqrt s}^\infty e^{-\frac{x^2}2} \D x
		\le \frac{2 \sqrt s}{\varepsilon \sqrt{2\pi}} e^{\frac{\varepsilon^2}{2s}}.
	\end{align}
	Stoga,
	\begin{equation}
		\frac 1u \alpha_{\varepsilon, T}(u) =
		\frac 1u \sup\limits_{s \le u, T} \P(\abs{X_s} > \varepsilon) \le
		\sup\limits_{s \le u, T} \frac 1s \P(\abs{X_s} > \varepsilon) \un, \quad u \downarrow 0,
	\end{equation}
	čime smo dokazali~\eqref{markov-dov2}.
\end{proof}

\end{document}
