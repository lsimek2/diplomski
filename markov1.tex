\documentclass[main.tex]{subfiles}

\begin{document}
\nocite{*}

\section{Markovljevi procesi i modifikacije} \label{sec:sp-markov}
U ovom odjeljku uvodimo Markovljeve procese. Pokazuje se da
su oni generalizacija \levy jevih i aditivnih procesa, tj.\ \levy jevi
i aditivni procesi zadovoljavaju Markovljevo svojstvo. Zanima nas ima
li \levy jev po distribuciji proces iz teorema~\ref{bddlpd}, t.~\ref{bddlpd2}, modifikaciju koja je \levy jev proces. Pitanje modifikacija
s \cadlag \ ili neprekidnim trajektorijama promatrat ćemo na Markovljevim procesima; dovoljne uvjete
dajemo u teoremu~\ref{markov-dov}.

Najprije uvedimo pojam tranzicijske funkcije. Intuitivno, funkcija mjeri
vjerojatnost prijelaza iz točke \( x \) u skup \( B \) između trenutaka \( s \) i \( t \).
\begin{definicija} \label{def:transf}
	Preslikavanje \( \Ptrans_{s, t}(x, B) \) gdje \( x \in \R^d \), \( B \in \mathcal B(\R^d) \)
	i \( 0 \le s \le t < \infty \) je \emph{tranzicijska funkcija} na \( \R^d \) ako
	\begin{enumerate}[label=(\roman*)]
		\item za fiksni \( x \) je \( B \mapsto \Ptrans_{s,t}(x, B) \) vjerojatnosna mjera na \( \mathcal B(\R^d) \),
		\item za fiksni \( B \) je \( x \mapsto \Ptrans_{s, t}(x, B) \) Borel-izmjeriva,
		\item vrijedi \( \Ptrans_{s,s}(x, B) = \delta_x(B) \) i
		\item zadovoljava relaciju
		      \begin{equation} \label{eq:chapkolm}
			      \int_{\R^d} \Ptrans_{s, t}(x, \D y) \Ptrans_{t, u} (y, B) = \Ptrans_{s, u}(x, B), \quad u \ge t.
		      \end{equation}
	\end{enumerate}
	Dodatno, razmatramo uvjete:
	\begin{enumerate}[label=(\roman*), resume]
		\item \( \Ptrans_{s+h, t+h}(x, B) \) ne ovisi o \( h \) i  \label{deftransf-star1}
		\item \( \Ptrans_{s, t}(x, B) = \Ptrans_{s,t}(0, B-x) \). \label{deftransf-star2}
	\end{enumerate}
	Ako vrijedi~\ref{deftransf-star1} kažemo da je tranzicijska funkcija \emph{vremenski homogena}.
	Ako vrijedi~\ref{deftransf-star2} kažemo da je \emph{prostorno homogena} ili \emph{invarijantna na translaciju}.
\end{definicija}

\begin{komentar}
	\begin{itemize}
		\item Jednakost~\eqref{eq:chapkolm} zove se identitet Chapman--Kolmogorova.
		      % ovdje ono o izmjerivosti?
		\item Ako familiju \( \{\Ptrans_i{s, t} \st 0 \le s \le t \} \) čine vremenski homogene tranzicijske funkcije, označavamo ju s \( \left\{ \Ptrans_t \st t \ge 0 \right\} \) gdje \( \Ptrans_t = \Ptrans_{0, t} \).
	\end{itemize}
\end{komentar}
\begin{komentar} \label{markov-konstr}
	Za familiju \( \left\{ \Ptrans_{s, t} \right\} \) tranzicijskih funkcija i početnu vrijednost \( a \in \R^d \) možemo konstruirati proces \( \left\{ Y_t \right\} \) kako slijedi. Tehnika i ideje slične su
	kao u dokazu teorema~\ref{bddlpd} pa ćemo biti manje detaljni. Definiramo na uobičajeni način \( \Omega \), \( \mathcal F \) i \( \left\{ Y_t \right\} \). Za \( 0 \le t_1 < \cdots < t_n \) i \( B_0, \ldots, B_n \in \mathcal B(\R^d) \) definiramo vjerojatnosnu mjeru preko
	\begin{equation} \label{markov-konstr-mjera}
		\begin{aligned}
			\mu_{t_0, \ldots, t_n}^{(a)}(B_0  \times \cdots \times B_n)                       \\
			= \int_{\R^d} \cdots \int_{\R^d} & 1_{B_0}(x_0)\cdots 1_{B_n}(x_n)                \\
			                                 & \Ptrans_{t_{n-1}, t_n}(x_{n-1}, \D x_n) \cdots
			\Ptrans_{t_0, t_1}(x_0, \D x_1) \Ptrans_{0, t_0}(a, \D x_0).
		\end{aligned}
	\end{equation}
	Suglasnost takve familije slijedi iz~\eqref{eq:chapkolm} te se ona proširi na jedinstvenu vjerojatnosnu mjeru \( \Ptrans^{a} \) na \( \mathcal F \).

	Slična konstrukcija moguća je za drukčije početno vrijeme \( s > 0 \). Tada \( \Omega^s = (\R^d)^{\left[ s, \infty \right\rangle} \) i definiramo proces \( \left\{ Y_t \st t \ge s \right\} \).
	Familija mjera \[ \left\{ \mu^{(s,a)}_{t_0,\ldots,t_n} \st s \le t_0 < \cdots < t_n \right\} \]
	definira se analogno kao u~\eqref{markov-konstr-mjera} i proširi se na jedinstvenu vjerojatnosnu mjeru \( \P^{s,a} \) na \( \sigma \)-algebri \( \mathcal F^s \).
	Možemo iste oznake s \( s=0 \) koristiti za objekte iz prošlog paragrafa.
\end{komentar}

\begin{definicija}
	Slučajni proces \( \left\{ X_t \right\} \) definiran na vjerojatnosnom prostoru \( \left( \Omega, \mathcal F, \P \right) \) je \emph{Markovljev proces sa familijom tranzicijskih funkcija} \( \left\{ \Ptrans_{s, t} \right\} \) \emph{i početnom vrijednosti} \( a \) ako je po distribuciji jednak procesu \( \left\{ Y_t \st t \ge 0 \right\} \) definiranom u komentaru~\ref{markov-konstr}. Ako su tranzicijske funkcije vremenski homogene, i sam proces naziva se \emph{vremenski homogenim}. Analogno, za proces
	\( \left\{ X_t \st t \ge s \right\} \) jednak po distribuciji procesu \( \left\{ Y_t \st t \ge s\right\} \)
	na prostoru \( \left( \Omega^s, \mathcal F^s, \P^{s,a} \right) \) može se dodati da ima \emph{početno vrijeme} \( s \). Procesi \( \left\{ Y_t \right\} \) zovu se \emph{trajektorijska reprezentacija}.
\end{definicija}

Vezu Markovljevih s aditivnim i \levy jevim procesima možemo ukratko i pomalo neprecizno
(v.~\cite[\textsection 10]{sato}) rezimirati ovako:
\begin{itemize}
	\item aditivni (po distribuciji) proces odgovara Markovljevom procesu s prostorno homogenom tranzicijskom funkcijom i
	\item \levy jev (po distribuciji) proces odgovara Markovljevom procesu s prostorno i vremenski homogenom tranzicijskom funkcijom.
\end{itemize}

Sljedeća propozicija uvodi varijantu Markovljevog svojstva za ovu vrstu procesa.
Većina dokaza je tehnička i temelji se na Lebesgueovoj indukciji pa ga izostavljamo,
ali neke stvari pojašnjavamo u komentaru~\ref{markov-prop-kom}.

\begin{propozicija}
	Neka je \( \left\{ Y_t \right\} \) trajektorijska reprezentacija Markovljevog procesa s
	familijom tranzicijskih funkcija \( \left\{ \Ptrans_{s, t}  \right\} \). Neka su
	\( 0 \le t_0 < \cdots < t_n \) i \( f \) ograničena Borel-izmjeriva funkcija na \( \R^d \).
	S \( \E^{s,a} \) označimo matematičko očekivanjem s obzirom na vjerojatnosnu mjeru
	\( \P^{s,a} \). Tada je \( a \mapsto \E^{0,a}[f(Y_{t_0},\ldots,Y_{t_n})] \) Borel-izmjeriva za sve \( a \in \R^d \) i
	\begin{equation} \label{markov-prop1}
		\begin{aligned}
			\E^{0,a}[f(Y_{t_0},\ldots,Y_{t_n})]                                               \\
			= \int_{\R^d} \cdots \int_{\R^d} & f(x_0,\ldots,x_n)                              \\
			                                 & \Ptrans_{t_{n-1}, t_n}(x_{n-1}, \D x_n) \cdots
			\Ptrans_{t_0, t_1}(x_0, \D x_1) \Ptrans_{0, t_0}(a, \D x_0).
		\end{aligned}
	\end{equation}
	Nadalje, ako su \( 0 \le s_0 < \cdots < s_m \le s \) i \( g \) ograničena Borel-izmjeriva
	funkcija na \( \R^d \) vrijedi\footnote{\( \E^{s,Y_s}[\cdot] \) znači \( h(Y_s) \) gdje
		je \( h(x) = \E^{s,x}[\cdot] \)}
	\begin{equation} \label{markov-svojstvo}
		\begin{aligned}
			\E^{0,a}[ & g(Y_{s_0},\ldots,Y_{s_m}) f(Y_{s+t_0},\ldots,Y_{s+t_n})]                                   \\
			          & =\E^{0,a}\bigl[g(Y_{s_0},\ldots,Y_{s_m}) \E^{s, Y_s}[f(Y_{s+t_0},\ldots,Y_{s+t_n})]\bigr].
		\end{aligned}
	\end{equation}
\end{propozicija}

\begin{komentar} \label{markov-prop-kom}
	Relacija~\eqref{markov-svojstvo} zove se Markovljevo svojstvo. Objasnimo prvo bazu Lebesgueove
	indukcije u dokazu~\eqref{markov-svojstvo}. Uzimamo
	\[
		f = \prod_{k=0}^n 1_{B_k} \quad \mathrm i \quad g = \prod_{k=0}^m 1_{C_k},
	\]
	gdje su \( B_0, \ldots B_n, C_0, \ldots, C_m \) Borelovi. Raspišemo li lijevu stranu od~\eqref{markov-svojstvo} preko~\eqref{markov-prop1}
	dobivamo da je jednaka
	\begin{equation} \label{Pbaza}
		\P^{0,a}(Y_{s_0} \in C_0, \ldots, Y_{s_m} \in C_m, Y_{s+t_0} \in B_0, \ldots, Y_{s+t_n} \in B_n).
	\end{equation}
	zbog~\eqref{markov-konstr-mjera}. U~\eqref{Pbaza} dodamo još \( Y_s \in \R^d \) prije nego raspišemo po~\eqref{markov-konstr-mjera}. Unutarnjih \( n + 1 \) integrala je jednako
	\begin{equation}
		\begin{aligned}
			\int_{\R^d} \int_{\R^d} \cdots \int_{\R^d} & \prod_{k=0}^n 1_{B_k}(y_k) \Ptrans_{s+t_{n-1}, s+t_n} (y_{n-1}, \D y_n)      \\
			                                           & \cdots \Ptrans_{s+t_0, s+t_1}(y_0, \D y_1) \Ptrans_{s, s+t_0} (y_s, \D y_0),
		\end{aligned}
	\end{equation}
	što po~\eqref{markov-prop1} i~\eqref{markov-konstr-mjera} za opći \( s \) prepoznajemo kao \( \E^{s, y_s}[f(Y_{s+t_0},\ldots,Y_{s+t_n})] \).
	Uvrštavanjem natrag u raspis opet primijenimo~\eqref{markov-prop1} da zaključimo da je upravo riječ o desnoj strani u~\eqref{markov-svojstvo}.

	Intuitivno, Markovljevo svojstvo govori da budućnost ovisi samo o trenutnom stanju procesa, a ne i o prošlima. Preciznije, za predvidjeti budućnost jednako je korisna
	sadašnjost koliko i svi dosad poznati događaji. Pokažimo kako to doista slijedi iz~\eqref{markov-svojstvo}. Fiksirajmo \( s, t_0, \ldots, t_n \), uvedimo
	pokrate \( X = f\left(Y_{s+t_0},\ldots,Y_{s+t_n}\right) \) i \( h(x) = \E^{s,x}(X)  \) te definirajmo \( \sigma \)-algebru
	\[
		\mathcal F_s = \sigma \left( Y_t \st 0 \le t \le s \right).
	\]
	Razmotrimo i skup
	\begin{equation} \label{markov-G}
		G = \left\{ g(Y_{s_0},\ldots,Y_{s_m}) \st 0 \le s_0 < \cdots < s_m \le s, g \mathrm{\ Borelova \ i \ ograni\check cena} \right\}.
	\end{equation}
	Kad bismo u~\eqref{markov-G} izostavili uvjet da je \( g \) ograničena bio bi \( G \) jednak skupu svih \( \mathcal F_s \)-izmjerivih varijabli (v.~\cite[tm.~8.6]{sarapa}).
	Ipak, budući da svaku \( g \) možemo dobiti kao limes ograničenih, lako je dokazati da \( \sigma(G) = \mathcal F_s \). U novim oznakama~\eqref{markov-svojstvo}
	postaje
	\begin{equation}
		\E^{0,a}[ZX] = \E^{0,a}[Zh(Y_s)], \quad Z \in G,
	\end{equation}
	što se istim principom proširuje na sve \( Z \) izmjerive u \( \mathcal F_s \). Budući da je \( h(Y_s) \) očito izmjeriva u \( \mathcal F_s \), po definiciji uvjetnog očekivanja
	(v.~\cite[str.~579]{sarapa}) zaključujemo da je \( h(Y_s) = \E^{0,a}[X \given \mathcal F_s] \). No, \( h(Y_s) \) je izmjeriva i u \( \sigma(Y_s) \) te zbog
	\( \sigma(Y_s) \subseteq \mathcal F_s \) opet po definiciji \( h(Y_s) = \E^{0,a}[X \given Y_s] \). Dakle,
	\begin{equation}
		\E^{0,a}\left[ X \given \mathcal F_s \right] = \E^{0,a}[X \given Y_s].
	\end{equation}
	Analogno prethodnoj diskusiji, \( X \) se može zamijeniti bilo kojom varijablom izmjerivom u \( \sigma(Y_t \st t \ge s) \). Zbog te općenitosti dobivamo i
	\begin{equation}
		\P^{0,a}\left[ X \in A \given \mathcal F_s \right] = \P^{0,a}\left[ X \in A \given Y_s \right], \quad A \in \mathcal B(\R^d).
	\end{equation}
\end{komentar}

\end{document}
