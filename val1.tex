\documentclass[main.tex]{subfiles}
\newcommand{\mr}[1]{\mathrm{#1}}
\newcommand{\mc}[1]{\mathcal{#1}}

\begin{document}
\section{Osnovno o valićima. Haarov sistem}\label{sec:val1}
Nakon Fourierove transformacije, javlja se ideja funkcije dekomponirati
s obzirom na neku drugu familiju funkcija. Unatoč brojnim lijepim svojstvima
Fourierove transformacije koja smo vidjeli u prošlom odjeljku, možemo naći i dvije
mane. Prvo: da daje globalno ponašanje funkcije u frekvencijskoj dimenziji, ali
nije osjetljiva na lokalne fenomene. Drugo, da iako su funkcije
\( x \mapsto e^{i\xi x} \) jednostavne u frekvencijskoj dimenziji,
u vremenskoj nemaju kompaktan nosač, a nisu niti integrabilne (osim u smislu temperiranih distribucija).
To nas može motivirati da ih zamijenimo funkcijama koje su po tom pitanju bolje.
Pokazuje se da je tu uvijek nužan kompromis, tj.\ inzistiranjem na novim svojstvima
izgubit ćemo neka lijepa svojstva koja smo prije imali.

Jedna takva alternativna familija su \emph{valići}.
Počnimo od neke funkcije \( \psi \colon \R \to \C \)
koju zovemo \emph{matični valić}.
Zatim, za \( a \in \R \setminus \left\{ 0 \right\} \)
i \( b \in \R \) definiramo
\begin{equation}
	\psi_{a,b}(x) = \abs a^{-1/2} \psi \left( \frac {x-b}a \right), \quad x \in \R.
\end{equation}
Primijetimo da s \( b \) pomičemo početni \( \psi \) po \( x \)-osi, dok smanjenjem \( a \)
sužavamo \( \psi \) i povećavamo amplitudu. Riječ je redom o translaciji i dilataciji,
pa u oznakama komentara~\ref{kom:four-kovac} možemo pisati i
\( \psi_{a,b} = \mr D_a \mr T_b \psi \). Ta dva operatora onda generiraju
\emph{sistem} funkcija pa je pravilnije govoriti o valićima kao sistemu funkcija,
dok se za neki konkretni matični valić \( \psi \) dobivamo \emph{familiju} \( \left\{ \psi_{a,b} \right\}_{a\neq 0, b \in \R} \).

Pitanje je kakva svojstva mora imati \( \psi \). Zahtjevi se razlikuju ovisno o primjeni,
ali vjerojatno najčešći i najosnovniji zahtjev je
\begin{equation}\label{eq:psi-dopustiv}
	\int_\R \psi(x) \D x = 0.
\end{equation}
Matični valić \( \psi \) koji zadovoljava~\eqref{eq:psi-dopustiv} ponekad se zove \emph{dopustiv}.
Objasnimo taj uvjet. Uzmimo svakako \( \psi \in \L^2(\R) \) tako da ima
Fourierovu transformaciju. Definiramo \emph{neprekidnu valićnu transformaciju} \( {\mc T^{\mr{wav}}} \) s
\begin{equation}\label{eq:cwt}
	(\mc T^{\mr{wav}} f)(a,b) = \skp f{\psi_{a,b}} =
	\int_\R f(x) \abs a^{-1/2} \ol{\psi\left( \frac{x-b}a \right)} \D x, \quad f \in \L^2(\R).
\end{equation}
Riječ je o valićnom analogonu formule~\eqref{eq:four}.
U ovom radu se nećemo baviti neprekidnom valićnom transformacijom, pa ćemo
samo navesti i analogon formule inverzije~\eqref{eq:four-inverzija}:
\begin{equation}\label{eq:wav-inverzija}
	f = C_\psi^{-1} \int_\R \int_\R
	(\mc T^{\mr{wav}} f)(a,b) \psi_{a,b} \frac{\D a \D b}{a^2}, \quad f \in \L^2(\R),
\end{equation}
pri čemu je \( C_\psi \) konstanta koja mora biti konačna:
\begin{equation}\label{eq:cpsikon}
	C_\psi = 2\pi \int_\R \abs \xi ^{-1} \abs{\wh \psi(\xi)}^2 \D \xi < \infty.
\end{equation}
Ako je \( \psi \in \L^1(\R) \) je \( \wh \psi \)
neprekidan pa je jedini način da vrijedi~\eqref{eq:cpsikon}
da je \( \wh \psi (0)=0 \), a to je
upravo~\eqref{eq:psi-dopustiv} po definiciji~\ref{def:four}.
S druge strane, može se pokazati (v.~\cite[\textsection 2.4]{daub}) da ako vrijedi~\eqref{eq:psi-dopustiv}
i jači uvjet od integrabilnosti
\begin{equation}\label{eq:psijaceodl1}
	\int_\R \left( 1 + \abs x \right)^\alpha \psi(x) \D x < \infty
\end{equation}
za neki \( \alpha > 0 \) da slijedi~\eqref{eq:cpsikon}.
Uvjeti~\eqref{eq:cpsikon} i~\eqref{eq:psi-dopustiv} su u
praksi ekvivalentni jer će se zahtjevi na \( \psi \)
ipak biti dosta jači nego~\eqref{eq:psijaceodl1}.

Ne moramo \( a \) i \( b \) uzimati neprekidno, već je moguće
razmatrati samo diskretnu rešetku. Za \( j, k \in \Z \) definirajmo
\begin{equation}\label{eq:psijk}
	\psi_{j, k}(x) = 2^{-j/2} \psi(2^{-j}x-k), \quad x \in \R,
\end{equation}
tj.\ \( \psi_{j,k} = \mr D_{2^j}T_k \psi \). Familija
\( \left\{ \psi_{j,k} \right\}_{j,k\in\Z} \) je prebrojiva, a
idealno bi bilo kad bi bila ortonormirana baza prostora \( \L^2(\R) \).
Nije \emph{a priori} jasno da je to ikada moguće, ali
pokazat ćemo u drugom dijelu ovog i sljedećem odjeljku da
zapravo postoji puno takvih ortonormiranih baza.
To je prednost u odnosu na Fourierovu transformaciju jer
\( x \mapsto e^{ikx} \) za \( k \in \Z \) razapinju
samo periodičke funkcije. Odsad se bavimo familijama danima s~\eqref{eq:psijk}.

Prije nego damo primjer ortonormirane valićne baze,
vratimo se na pitanje svojstava \( \psi \).
Tipična poželjna svojstva koja se traže su sljedeća:
\begin{itemize}
	\item glatkoća, ili, blisko tome, nestajući momenti,
	\item kompaktan nosač u vremenskoj ili frekvencijskoj dimenziji (tj.\ kompaktnost od \( \supp \psi \) i \( \supp \wh \psi \) redom),
	\item kada nemamo kompaktnost da \( \psi \), neke derivacije od \( \psi \) ili \( \wh \psi \) padaju nekom brzinom,
	\item da je \( \psi  \) parna funkcija (simetrična; ekvivalentno je da je \( \wh \psi \) realna) ili da je
	      realan; alternativno, u nekim primjenama korisni su kompleksni valići.
\end{itemize}
Sljedeći teorem (v.~\cite[\textsection 9]{daub}) pokazuje da su uz vrlo razumne zahtjeve ortonormirane baze na
\( \L^2(\R) \) ujedno bezuvjetne baze na drugim funkcijskim prostorima.

\newcommand{\fusnota}{\footnote{oznaka \( \lesssim \) u \( f(x) \lesssim g(x) \) znači \( f(x) \le Cg(x) \) za neku konstantu \( C > 0 \) neovisnu o \( x \)}}
\begin{teorem} \label{bezbazaLp}
	Ako je \( \psi \) takav da je \( \left\{ \psi_{j,k} \st j,k \in \Z \right\} \) iz~\eqref{eq:psijk} ortonormirana bazu na \( \L^2(\R) \) i ako dodatno vrijede uvjeti\fusnota
	\begin{equation} \label{dovuvjetipsi}
		\psi \in \mr C^1(\R) \qquad \mathrm i \qquad \abs{\psi(x)}, \abs{\psi'(x)} \lesssim (1 + \abs x)^{-1-\varepsilon},
	\end{equation}
	tada je \( \left\{ \psi_{j,k} \st j,k \in \mathbb Z \right\} \) također bezuvjetna baza na \( \L^p(\R) \), \( 1 < p < \infty \).
\end{teorem}
Zapravo, isti uvjeti dovoljni su da \( \psi_{j,k} \) čine ortonormiranu bazu \emph{Hardy--Littlewoodovog}
prostora kojeg ovdje nećemo definirati, a uz nešto jače uvjete i drugih funkcijskih prostora kao što
su npr.\ \emph{Soboljevljevi prostori}.

Poželjna svojstva imaju svoje granice. Najpoznatiji
primjer je \emph{princip neodređenosti} koji govori
da nije moguće (osim u trivijalnom slučaju)
postići da funkcija ima kompaktan nosač i u
vremenskoj i u frekvencijskoj dimenziji (v.~\cite[theorem~4]{kovac-wroclaw}).

\begin{teorem}
	Ako je \( f \in \L^2(\R) \) takva da su i \( \supp f \)
	i \( \supp \wh f \) kompaktni, tada je \( f=0 \)~g.s.
\end{teorem}
Sljedeći teorem (v.~\cite{daub2}) uvodi Daubechiesjine valiće, ali i sugerira ograničenje
da ne postoje valići klase \( \mr C^\infty \) kompaktnog nosača
koji čine ortonormiranu bazu na \( \L^2(\R) \).
\begin{teorem}
	Za svaki \( r \in \N \) postoji \( \psi \in \mr C_\mrc ^r(\R) \)
	takav da je s~\eqref{eq:psijk} dana
	ortonormirana baza na \( \L^2(\R) \).
\end{teorem}
Poznato je i što smo rekli, da tvrdnja ne vrijedi za \( r = \infty \).
Meyerovi valići koje uvodimo u odjeljku~\ref{sec:meyer}
nalazit će se taman izvan ovih ograničenja.

\bigskip
U nastavku odjeljka cilj nam je dokazati da Haarovi valići čine ortonormiranu bazu
na \( \L^2(\R) \).

\end{document}
